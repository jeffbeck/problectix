%%%%%%%%%%%%%%%%%%%%%%%%%%%%%%%%%%%%%%%%%%%%%%%%%%%%%%%%%%%%%%%%%%%%%%%%
%% $Id$
%%%%%%%%%%%%%%%%%%%%%%%%%%%%%%%%%%%%%%%%%%%%%%%%%%%%%%%%%%%%%%%%%%%%%%%%
\def\DTK@RCS$#1: #2 #3${#2}
\def\filename{cwpuzzle.dtx}
\xdef\fileversion{\DTK@RCS$Revision$}
\xdef\filedate{\DTK@RCS$Date$}
\let\docversion=\fileversion
\let\docdate=\filedate
%%%%%%%%%%%%%%%%%%%%%%%%%%%%%%%%%%%%%%%%%%%%%%%%%%%%%%%%%%%%%%%%%%%%%%%%
%\iffalse
\typeout{%
%% Purpose:
        Package: cwpuzzle \filedate\space\fileversion\space
        for typesetting crossword puzzles.
}% 
%%
%% Documentation:
%%      The documentation  can be generated   from the original  file
%%      cwpuzzle.dtx  with  the  doc  style/package.   LaTeX the file 
%%	cwpuzzle.dtx to get the full documentation in dvi format.
%%
%%
%% Author: Gerd Neugebauer
%%         Mainzer Str. 8
%%         56321 Rhens
%% Mail:   gerd@informatik.uni-koblenz.de
%%
%% Copyright (C) 1996 Gerd Neugebauer
%%
%%  cwpuzzle.dtx  is distributed in  the hope that  it will be useful,
%%  but  WITHOUT  ANY WARRANTY.    No  author or  distributor  accepts
%%  responsibility to  anyone for the  consequences of using it or for
%%  whether  it serves any particular purpose  or works at all, unless
%%  he says so in writing.
%% 
%%  Everyone  is granted permission  to copy,  modify and redistribute
%%  dtk.dtx,  provided this  copyright  notice  is preserved  and  any
%%  modifications are indicated.
%% 
%%
%%  This style is still under development and  may be replaced with a
%%  new version which provides an enhanced functionality.
%%
%<*driver>
\documentclass{ltxdoc}
\usepackage{cwpuzzle}
\EnableCrossrefs
\CodelineIndex
\RecordChanges
\setcounter{IndexColumns}2
\begin{document}
  \DocInput{cwpuzzle.dtx}
  \newpage
  \PrintChanges
\end{document}
%</driver>
%\fi
%
% \title{A \LaTeX\ Package for Typesetting\\ Crossword
%        Puzzles\thanks{This file documents \filename\ version 
%        \fileversion\ as of \filedate.}} 
% \author{Gerd Neugebauer\\
%	  Mainzer Stra\ss{}e 8\\
%	  56321 Rhens (Germany)\\
%	  Net: {\tt gerd@informatik.uni-koblenz.de}}
%
% \date{{\footnotesize Documentation date: \docdate}}
%
% \maketitle
%
%%%%%%%%%%%%%%%%%%%%%%%%%%%%%%%%%%%%%%%%%%%%%%%%%%%%%%%%%%%%%%%%%%%%%%%%
%%%%%%%%%%%%%%%%%%%%%%%%%%%%%%%%%%%%%%%%%%%%%%%%%%%%%%%%%%%%%%%%%%%%%%%%
%
% \DoNotIndex{\ ,\",\',\.,\[,\\,\],\^,\`,\~,\@,\@dottedtocline}
% \DoNotIndex{\@empty,\@nameuse,\advance,\begin,\begingroup}
% \DoNotIndex{\catcode,\def,\else,\end,\endgroup,\fi,\filedate}
% \DoNotIndex{\footnotesize,\framebox,\gdef,\hfill,\if,\ifx,\let,\makebox}
% \DoNotIndex{\mbox,\newcommand,\newcount,\newdimen,\newenvironment}
% \DoNotIndex{\newif,\noindent,\normalsize,\null,\par,\put}
% \DoNotIndex{\relax,\rm,\rule,\scriptsize,\sf,\textsf,\textwidth}
% \DoNotIndex{\the,\unitlength,\uppercase,\xdef,\ProvidesPackage}
%
%%%%%%%%%%%%%%%%%%%%%%%%%%%%%%%%%%%%%%%%%%%%%%%%%%%%%%%%%%%%%%%%%%%%%%%%
%
%    \changes{1.3}{1996/10/28}{First public release.}
%
%%%%%%%%%%%%%%%%%%%%%%%%%%%%%%%%%%%%%%%%%%%%%%%%%%%%%%%%%%%%%%%%%%%%%%%%
%    \CheckSum{284}
%%%%%%%%%%%%%%%%%%%%%%%%%%%%%%%%%%%%%%%%%%%%%%%%%%%%%%%%%%%%%%%%%%%%%%%%
%%  \CharacterTable
%%  {Upper-case    \A\B\C\D\E\F\G\H\I\J\K\L\M\N\O\P\Q\R\S\T\U\V\W\X\Y\Z
%%   Lower-case    \a\b\c\d\e\f\g\h\i\j\k\l\m\n\o\p\q\r\s\t\u\v\w\x\y\z
%%   Digits        \0\1\2\3\4\5\6\7\8\9
%%   Exclamation   \!     Double quote  \"     Hash (number) \#
%%   Dollar        \$     Percent       \%     Ampersand     \&
%%   Acute accent  \'     Left paren    \(     Right paren   \)
%%   Asterisk      \*     Plus          \+     Comma         \,
%%   Minus         \-     Point         \.     Solidus       \/
%%   Colon         \:     Semicolon     \;     Less than     \<
%%   Equals        \=     Greater than  \>     Question mark \?
%%   Commercial at \@     Left bracket  \[     Backslash     \\
%%   Right bracket \]     Circumflex    \^     Underscore    \_
%%   Grave accent  \`     Left brace    \{     Vertical bar  \|
%%   Right brace   \}     Tilde         \~}
%%
%%%%%%%%%%%%%%%%%%%%%%%%%%%%%%%%%%%%%%%%%%%%%%%%%%%%%%%%%%%%%%%%%%%%%%%%
%    \MakeShortVerb{|}
%
%    \begin{abstract}
%      cpwuzzle.dtx provides a package to typeset crossword
%      puzzles. There is an graphical user interface written in
%      Tcl/Tk which does most of the work in creating a file which is
%      conform with this style. Thus it is clear that the macros
%      provided are not tailored towards a human user.
%
%      The package can be used to produce several types of puzzles like
%      the classical crossword puzzle, a number puzzle, and fill-in
%      puzzles. 
%    \end{abstract}
%
%    \tableofcontents
%    \newpage
%
%    \section{About Crossword Puzzles}
%
%    Crossword puzzles are can be a an amusing but also a challenging
%    hobby. Unfortunately at the time of this writing I am not aware
%    of any good package to typeset crossword puzzles with \LaTeX.
%    Thus I decided to make one which at least fits my needs. 
%
%    There are several types of crossword puzzles aroung. This
%    package can only be used to typeset several of them. The basic
%    assumption in this package is that puzzles are rectangular
%    arragnements of boxes. Some of these boxes are black and others
%    are prepared to take single letters.
%    Each word in the grid is enclosed in black boxes or the outside.
%
%    Optionally there may be rectangular regions left blank inside the
%    puzzle. They can be used to place ads or other informative texts
%    inside the puzzle.
%
%    \subsection{Classical Crossword Puzzles}\label{sec:classical}
%
%    \PuzzleUnsolved
%    \begin{minipage}{.35\textwidth}%
%      \begin{Puzzle}{5}{3}%
%        |*   |*   |[1]E|X  |*   |.
%        |[2]A|[3]S|T   |*  |[4]T|.
%        |*   |[5]P|A   |R  |T   |.
%      \end{Puzzle}%
%    \end{minipage}%
%    \begin{minipage}{.6\textwidth}
%      \begin{PuzzleClues}{Across}%
%        \Clue{1}{EX}{unit of measure}%
%        \Clue{2}{AST}{\(\ast\)}%
%        \Clue{5}{PART}{sectioning unit}%
%      \end{PuzzleClues}%
%      \begin{PuzzleClues}{Down}%
%        \Clue{1}{ETA}{\(\eta\)}%
%        \Clue{3}{SP}{unit of measure}%
%        \Clue{4}{TT}{nonproportional font}%
%      \end{PuzzleClues}%
%    \end{minipage}\medskip
%
%    The ``classical'' type of a crossword puzzle words are marked
%    with numbers and each word is acomapanied with a clue which
%    should help (or confuse) the reader. Those clues are listed after
%    the frame of the puzzle.
%
%    \subsection{Number Crossword Puzzles}\label{sec:numbered}
%
%    \begin{minipage}{.35\textwidth}%
%      \begin{Puzzle}{5}{3}%
%        |*   |*   |[6]E|[1]X|*   |.
%        |[5]A|[2]S|[7]T|*   |[7]T|.
%        |*   |[4]P|[5]A|[3]R|[7]T|.
%      \end{Puzzle}%
%    \end{minipage}%
%    \begin{minipage}{.6\textwidth}
%      \PuzzleLetters{AEPRSTX}\medskip\par 
%      \PuzzleNumbers{XSRPAET}%
%    \end{minipage}\medskip
%
%    The ``number puzzle'' variant contains only numbers instead of
%    letters. Different numbers denote different letters. There are no
%    clues. The reader is assumed to find a complete list of letters
%    by filling appropriate words into the grid. Sometimes a word is
%    already entered into the grid to ease the start.
%
%    \subsection{Fill-In Crossword Puzzles}\label{sec:fill-in}
%
%    \begin{minipage}{.35\textwidth}%
%      \begin{Puzzle}{5}{3}%
%        |*  |*  |E  |X  |*  |.
%        |A  |S  |T  |*  |T  |.
%        |*  |P  |A  |R  |T  |.
%      \end{Puzzle}%
%    \end{minipage}%
%    \begin{minipage}{.6\textwidth}
%      \begin{PuzzleWords}{2}
%        \Word{EX}%
%        \Word{SP}%
%        \Word{TT}%
%      \end{PuzzleWords}%
%      \begin{PuzzleWords}{3}
%        \Word{AST}%
%        \Word{ETA}%
%      \end{PuzzleWords}%
%      \begin{PuzzleWords}{4}
%        \Word{PART}%
%      \end{PuzzleWords}%
%    \end{minipage}\medskip
%
%    The ``fill-in puzzle'' variant consists of a frame containing
%    only black and white boxes. Additionally a list of words is given
%    which have to be put into the frame until none is left and the
%    frame is completed.
%
%    \subsection{Solutions}
%
%    \PuzzleSolution
%    \begin{minipage}{.35\textwidth}%
%      \begin{Puzzle}{5}{3}%
%        |*   |*   |[1]E|X  |*   |.
%        |[2]A|[3]S|T   |*  |[4]T|.
%        |*   |[5]P|A   |R  |T   |.
%      \end{Puzzle}%
%    \end{minipage}%
%    \begin{minipage}{.6\textwidth}
%    \end{minipage}\medskip
%
%    Often it is not only desirable to typeset the unsolved crossword
%    puzzle but also the solution. This means that all the letters
%    have to be filled in. This should be possible with the same
%    source as the questions to avoid typos or redundancies leading to
%    additional work. 
%
%    Several variants of solutions come to mind. Primarily the
%    solution should show the letters and suppress any clues. One
%    major distinction is also wether or not the numbers of the words
%    should be shown in the solution as well.\medskip
%
%    \noindent%
%    \PuzzleSolution[true]%
%    \begin{minipage}{.35\textwidth}%
%      \begin{Puzzle}{5}{3}%
%        |*   |*   |[1]E|X  |*   |.
%        |[2]A|[3]S|T   |*  |[4]T|.
%        |*   |[5]P|A   |R  |T   |.
%      \end{Puzzle}%
%    \end{minipage}%
%    \begin{minipage}{.6\textwidth}
%    \end{minipage}%
%
%    Finally there are the lists of letters in numbered puzzles. In
%    the solution they will show the letters in them as well.\medskip 
%
%    \noindent%
%    \PuzzleSolution[false]%
%    \begin{minipage}{.35\textwidth}%
%      \begin{Puzzle}{5}{3}%
%        |*   |*   |[6]E|[1]X|*   |.
%        |[5]A|[2]S|[7]T|*   |[7]T|.
%        |*   |[4]P|[5]A|[3]R|[7]T|.
%      \end{Puzzle}%
%    \end{minipage}%
%    \begin{minipage}{.6\textwidth}
%      \PuzzleLetters{AEPRSTX}\medskip\par 
%      \PuzzleNumbers{XSRPAET}%
%    \end{minipage}
%
%
%    \section{Input of Crossword Puzzles}
%
%    The basic idea behind this package is that a crossword puzzle is
%    specified in a seperate file. The actual appearance of the puzzle
%    is controlled by several options. Thus it should be possible to
%    produce the unsolved and the solved puzzle from the same
%    source. Before we describe the various options we will have a
%    look at the basic environments and macros used to specified a
%    crossword puzzle.
%
%    \DescribeEnv{Puzzle}%
%    This package provides the environment |Puzzle| which typesets the
%    frame of a crossword puzzle. This environment takes two
%    arguments. These arguments are the number of columns and the
%    number of the rows of the puzzle. This means that essentially
%    only rectangular puzzles can be typeset. 
%
%    The example from section~\ref{sec:classical} has been entered as
%    follows: 
%    \begin{verbatim}
%    \begin{Puzzle}{5}{3}%
%      |*   |*   |[1]E|X  |*   |.
%      |[2]A|[3]S|T   |*  |[4]T|.
%      |*   |[5]P|A   |R  |T   |.
%    \end{Puzzle}
%    \end{verbatim}
%    In this example we can see that inside the |Puzzle| environment
%    there is one special character. This is the bar \verb/|/. This
%    bar is an activge character in \TeX. Thus you can think of it
%    like a macro.
%
%    The \verb/|/ macro takes two arguments. The first argument is
%    optional, i.e. enclosed in brackets if present. This optional
%    argument denotes the number for numbered boxes.
%
%    The second argument is either empty |{}| or it consists of a
%    single character. This argument describes the action to be
%    performed.
%
%    \begin{itemize}
%    \item If this argument is a letter then it is simply shown
%    in the solution and suppressed in the unsolved crossword puzzle.
%
%    \item If this argument is an asterisk |*| then a black box is
%    produced.
%
%    \item If this argument is a dot |.| then this marks the end of
%    the current row. The next box is typeset at the beginning of the
%    following row.
%
%    \item If this argument is empty |{}| then a white box is
%    typeset. This box does not contain a letter, nor does it have a
%    frame. This macro can be used to leave room for larger boxed with
%    ads. Alternatively this can be used to disable certain boxes to
%    make a non-rectangular crossword puzzle.
%    \end{itemize}
%    
%    \begin{verbatim}
%    \begin{Puzzle}{5}{5}
%      |{}  |{}  |[1]S|.
%      |{}  |[2]M|I   |[3]D|.
%      |[4]T|I   |M   |E   |S   |.
%      |{}  |[5]N|E   |G   |.
%      |{}  |{}  |Q   |.
%    \end{Puzzle}
%    \end{verbatim}
%    \PuzzleUnsolved
%    \begin{minipage}{.35\textwidth}%
%      \begin{Puzzle}{5}{5}%
%        |{}  |{}  |[1]S|.
%        |{}  |[2]M|I   |[3]D|.
%        |[4]T|I   |M   |E   |S   |.
%        |{}  |[5]N|E   |G   |.
%        |{}  |{}  |Q   |.
%      \end{Puzzle}%
%    \end{minipage}%
%    \begin{minipage}{.6\textwidth}
%      \begin{PuzzleClues}{\textbf{Across: }}%
%        \Clue{2}{MID}{\(\mid\)}%
%        \Clue{4}{TIMES}{\(\times\)}%
%        \Clue{5}{NEG}{\(\neg\)}%
%      \end{PuzzleClues}%
%      \begin{PuzzleClues}{\textbf{Down: }}%
%        \Clue{1}{SIMEQ}{\(\simeq\)}%
%        \Clue{2}{MIN}{loglike function}%
%        \Clue{3}{DEG}{loglike function}%
%      \end{PuzzleClues}%
%    \end{minipage}%
%
%    Note that whitespace is ignored after the arguments but not
%    between the bar and the arguments.
%
%    \DescribeMacro{\Frame}
%    The macro |\Frame| can be used to typeset ads or other text into
%    larger boxes inside the frame of the crossword puzzle. For this
%    purpose five arguments are required. The first two arguments are
%    used to specify the leower left corner of the frame. The lower
%    left corner has the coordinates 0,0 and the numbers increase
%    upwards and to the right.
%
%    The third argument is the width of the frame and the fourth
%    argument is the height of the frame measured in number of boxes.
%    Finally, the fifth argument contains the text to be typeset. Per
%    default it is typeset in a minipage of the appropriate width
%    centered horizontally and vertically.
%
%    \begin{verbatim}
%    \begin{Puzzle}{8}{6}
%      \Frame{2}{2}{4}{2}{\sf Crossword\\Puzzle}
%      |[1]E|*   |[2]N |U    |L   |[3]L |*   |[4]V|.
%      |[5]T|[6]R|I    |A    |N   |G    |[7]L|E   |.
%      |A   |U   |{}   |{}   |{}  |{}   |[8]C|C   |.
%      |*   |L   |{}   |{}   |{}  |{}   |E   |*   |.
%      |[9]B|E   |T    |[10]A|*   |[11]L|I   |M   |.
%      |F   |*   |[12]L|A    |B   |E    |L   |*   |.
%    \end{Puzzle}
%    \end{verbatim}
%
%    \PuzzleUnsolved\noindent
%    \begin{minipage}{.5\textwidth}\PuzzleUnitlength=18pt
%    \begin{Puzzle}{8}{6}%
%      \Frame{2}{2}{4}{2}{\sf Crossword\\Puzzle}%
%      |[1]E|*  |[2]N|U|L|[3]L|*  |[4]V|.
%      |[5]T|[6]R|I|A|N|G|[7]L|E|.
%      |A|U|{}|{}|{}|{}|[8]C|C|.
%      |*  |L|{}|{}|{}|{}|E|*  |.
%      |[9]B|E|T|[10]A|*  |[11]L|I|M|.
%      |F|*  |[12]L|A|B|E|L|*  |.
%    \end{Puzzle}%
%    \end{minipage}%
%    \begin{minipage}{.47\textwidth}
%    \begin{PuzzleClues}{\textbf{Across: }}%
%      \Clue{2}{NULL}{empty}%
%      \Clue{5}{TRIANGLE}{\(\triangle\)}%
%      \Clue{8}{CC}{carbon copy (letter.sty)}%
%      \Clue{9}{BETA}{\(\beta\)}%
%      \Clue{11}{LIM}{limes}%
%      \Clue{12}{LABEL}{mark it}%
%    \end{PuzzleClues}%
%    \begin{PuzzleClues}{\textbf{Down: }}%
%      \Clue{1}{ETA}{\(\eta\)}%
%      \Clue{2}{NI}{\(\ni\)}%
%      \Clue{3}{LG}{logarithm}%
%      \Clue{4}{VEC}{\(\vec{}\)}%
%      \Clue{6}{RULE}{black rectangle}%
%      \Clue{7}{LCEIL}{\(\lceil\)}%
%      \Clue{9}{BF}{bold face}%
%      \Clue{10}{AA}{\AA}%
%      \Clue{11}{LE}{\(\le\)}%
%    \end{PuzzleClues}%
%    \end{minipage}%
%
%
%    \DescribeEnv{PuzzleClues}
%    The clues in the classical crossword puzzle are typeset with the
%    use of the environment |PuzzleClues|. This environment takes one
%    argument which is typeset before the clues. The environment takes
%    roughly the half of the textwidth and make a minipage with this
%    width. Thus two invocations of this environment are typeset side
%    by side.
%
%    Alternatively if the solution is typeset then the environment
%    |PuzzleClues| has no effect.
%
%    \begin{verbatim}
%    \begin{PuzzleClues}{\textbf{Across}}%
%      \Clue{1}{EX}{unit of measure}%
%      \Clue{2}{AST}{\(\ast\)}%
%      \Clue{5}{PART}{sectioning unit}%
%    \end{PuzzleClues}%
%    \begin{PuzzleClues}{\textbf{Down}}%
%      \Clue{1}{ETA}{\(\eta\)}%
%      \Clue{3}{SP}{unit of measure}%
%      \Clue{4}{TT}{nonproportional font}%
%    \end{PuzzleClues}%
%    \end{verbatim}
%
%    \DescribeMacro{Clue}
%    The environment |PuzzleClues| defines one local macro. This macro
%    is named |\Clue| and takes three arguments. The first argument is
%    the number of the word. This should correspond to the number in
%    the puzzle frame. The second argument is the word itself.
%    Currently this not used at all. Finally the third argument is the
%    clue for the word. 
%
%    If the unsolved puzzle is typeset then the first and the third
%    argument are used. Otherwise all arguments are silently absorbed.
%
%    \DescribeMacro{\PuzzleLetters}
%    The macro |\PuzzleLetters| can be used to typeset the list of
%    used letters in numbered crossword puzzles. It has one argument
%    which are the used letters (preferably in alphabethical order.
%
%    \DescribeMacro{\PuzzleNumbers}
%    The macro |\PuzzleNumbers| can be used to generate a numbered
%    list of boxes for the numbered crossword puzzles. The user is
%    supposed to collect the found letters here.
%
%    \DescribeEnv{PuzzleWords}
%    The environment |PuzzleWords| can be sued to typeset the list of
%    words for a fill-in puzzle. It takes one argument. This is the
%    length of the words listed. For each length there should be an
%    invocation of this environment. The words in this environment are
%    supposed to be ordered alphabetically.
%
%    \DescribeMacro{\Word}
%    The macro |\Word| is defined inside the environment
%    |PuzzleWords|. It takes one argument which is the word itself.
%    
%    \begin{verbatim}
%    \begin{PuzzleWords}{2}
%      \Word{EX}%
%      \Word{SP}%
%      \Word{TT}%
%    \end{PuzzleWords}%
%    \begin{PuzzleWords}{3}
%      \Word{AST}%
%      \Word{ETA}%
%    \end{PuzzleWords}%
%    \begin{PuzzleWords}{4}
%      \Word{PART}%
%    \end{PuzzleWords}%
%    \end{verbatim}
%
%
%    \section{Parameters and Options}
%
%    \DescribeMacro{\PuzzleUnitlength}
%    The length |\PuzzleUnitlength| deternimes the width and height of
%    each single box in the frame of a crossword puzzle. The default
%    value is |20pt|.
%
%    \DescribeMacro{\PuzzleBlackBox}
%    The macro |\PuzzleBlackBox| contains the commands to produce the
%    black boxes. It has to produce at most of width and height of
%    |\PuzzleUnitlength|. Per default it just produces a black
%    rectangle of this size.
%
%    The following list shows some variants which can be achieved by
%    redefining the macro |\PuzzleBlackBox|.\medskip
%
%    \noindent
%    \unitlength=\PuzzleUnitlength
%    \framebox(1,1){\rule{.75\PuzzleUnitlength}{.75\PuzzleUnitlength}}
%    \hfill
%    \begin{minipage}{.9\textwidth}    
%    \begin{verbatim}
%    \renewcommand{\PuzzleBlackBox}{\rule{.75\PuzzleUnitlength}%
%                                        {.75\PuzzleUnitlength}}
%    \end{verbatim}
%    \end{minipage}
%
%    \noindent
%    \framebox(1,1){\framebox(.75,.75){\framebox(.5,.5){}}}
%    \hfill
%    \begin{minipage}{.9\textwidth}    
%    \begin{verbatim}
%    \renewcommand{\PuzzleBlackBox}{\framebox(.75,.75){%
%                                     \framebox(.5,.5){}}}
%    \end{verbatim}
%    \end{minipage}
%
%    Additional effects can be achieved by using shades of gray (with
%    the |graphics| package).
%
%    \DescribeMacro{\PuzzleFont}
%    The macro |\PuzzleFont| contains the font changing
%    command issued before the frame of the crosword puzzle.
%
%    \DescribeMacro{\PuzzleNumberFont}
%    The macro |\PuzzleNumberFont| contains the font changing
%    command issued before a number in the frame of the crosword
%    puzzle is typeset.
%
%    \DescribeMacro{\PuzzleClueFont}
%    The macro |\PuzzleClueFont| contains the font changing
%    command issued before the clues are typeset.
%
%    \DescribeMacro{\PuzzleWordsText}
%    The macro |\PuzzleWordsText| contains the text which is typeset
%    at the beginning of the environment |PuzzleWords|. It has one
%    argument which contains the length of the words listed.
%
%    \DescribeMacro{\PuzzleLettersText}
%    The macro |\PuzzleLettersText| contains the text which is typeset
%    at the beginning of the macro |PuzzleLetters|.
%
%    \DescribeMacro{\PuzzleSolution}
%    The macro |\PuzzleSolution| arranges everything that the
%    following puzzles are typeset in the ``solution'' mode, i.e. the
%    letters are shown and the clues are suppressed.
%
%    This macros has one optional argument which has to be |true| or
%    |false|. This argument determines whether or not the numbers
%    should also be shown in the solution. The default is |false|
%    which means that the numbers are suppressed in the solution.
%
%    \DescribeMacro{\PuzzleUnsolved}
%    The macro |\PuzzleUnsolved| arranges everything that the
%    following puzzles are typeset in the ``unsolved'' mode, i.e. the
%    letters are suppressed and the clues are shown.
%
%    \DescribeMacro{\PuzzleHook}
%    The macro |\PuzzleHook| is called at the end of the |Puzzle|
%    environment. It can be used to place additional graphical
%    elements in the puzzle frame.
%
%    The following example shows a crossword puzzle which we have seen
%    before and the definition for the |\PuzzleHook|.\medskip
%
%    \PuzzleUnsolved
%    \begin{minipage}{.35\textwidth}%
%      \newcommand\PuzzleHook{%
%        \put(0,2){\line(1,-1){2}}
%        \put(0,3){\line(1,1){2}}
%        \put(5,2){\line(-1,-1){2}}
%        \put(5,3){\line(-1,1){2}}
%      }
%      \begin{Puzzle}{5}{5}%
%        |{}  |{}  |[1]S|.
%        |{}  |[2]M|I   |[3]D|.
%        |[4]T|I   |M   |E   |S   |.
%        |{}  |[5]N|E   |G   |.
%        |{}  |{}  |Q   |.
%      \end{Puzzle}%
%    \end{minipage}%
%    \begin{minipage}{.6\textwidth}
%    \begin{verbatim}
%    \newcommand\PuzzleHook{
%      \put(0,2){\line(1,-1){2}}
%      \put(0,3){\line(1,1){2}}
%      \put(5,2){\line(-1,-1){2}}
%      \put(5,3){\line(-1,1){2}}
%    }
%    \end{verbatim}
%    \end{minipage}
%
%
%    \section{The Related Program}
%
%    There is a related program written in Tcl/Tk. This program can be
%    used to manually construct crosswrod puzzles and save them in a
%    format suitable for this package. Alternativly it can also store
%    the crossword puzzle prepared for |crosswrd.sty|.
%
%    Other features include the creation of a proper frame and filling
%    with words.
%
%    Right now I have not prepared a distribution of this program yet
%    since this program requires dictionaries which I can not
%    distribute legally.
%
%    The examples in this documentation have been computed with the
%    help of the |cwp| program.
%
%
%    \section{Further Plans}
%
%    Maybe I will add a mode for further variants of crossword puzzles
%    sometimes.
%
%
%
%    \StopEventually{}
%    \newpage
%
%
%    \section{The Implementation}
%
%    The crossword puzzle is basically implemented with the \LaTeX{}
%    picture environment. This gives us enough flexibility and
%    provides an high enough abraction such that we do not have to
%    fiddle around with to many low level details.
%
%    The natural unit in a crossword puzzle is a box which is empty
%    or black. Thus the |unitlength| is set to the width (and height)
%    of such a box. 
%
%
%    \subsection{Basic Definitions and Parameters}
%
%    First we identify this package.
%    
%    \begin{macrocode}
\ProvidesPackage{cwpuzzle}[\filedate gene]
%    \end{macrocode}
%
%    The dimen register |\PuzzleUnitlength| stores the height and
%    width of a box of the puzzle. The default is |20pt| which is
%    also shown in this documentation.
%    \begin{macrocode}
\newdimen\PuzzleUnitlength
\PuzzleUnitlength=20pt
%    \end{macrocode}
%    
%    \begin{macro}{\PuzzleClueFont}
%
%    The macro |\PuzzleClueFont| contains font changings
%    commands issued before the clues are typeset.
%    \begin{macrocode}
\newcommand\PuzzleClueFont{\footnotesize}
%    \end{macrocode}
%    \end{macro}
%    
%    \begin{macro}{\PuzzleFont}
%    
%    The macro |\PuzzleFont| contains font changings
%    commands issued before the puzzle is typeset.
%    \begin{macrocode}
\newcommand\PuzzleFont{\rm\normalsize}
%    \end{macrocode}
%    \end{macro}
%    
%    \begin{macro}{\PuzzleNumberFont}
%    The macro |\PuzzleNumberFont| contains font changings
%    commands issued before the numbers in a puzzle are typeset.
%    \begin{macrocode}
\newcommand\PuzzleNumberFont{\sf\scriptsize}
%    \end{macrocode}
%    \end{macro}
%    
%    \begin{macro}{\PuzzleHook}
%    Puzzles are typeset with the \LaTeX{} picture environment. At the
%    end of this environment the macro |\PuzzleHook| is
%    called. The package prodised an empty default. Users may want to
%    use this place to typeset additional elements on top of the
%    puzzle.
%
%    The puzzle uses a |\unitlength| of |\PuzzleUnitlength|. Thus it
%    is rather easy to adress the boxes in the puzzle.
%    
%    \begin{macrocode}
\let\PuzzleHook=\relax
%    \end{macrocode}
%    \end{macro}
%    
%    \subsection{The Frame of the Crossword Puzzle}
%    
%    To describe the coordinates where the next box should be typeset
%    we need two counters for the coordinates. These counters are now
%    allocated (even though we could use temporary counters from
%    \LaTeX). 
%    \begin{macrocode}
\newcount\Puzzle@X
\newcount\Puzzle@Y
%    \end{macrocode}
%    
%    \begin{macrocode}
\begingroup
\catcode`\|=13
\gdef\Puzzle@@solution{
  \let|=\Puzzle@Box@@solution
  \let\Frame=\Puzzle@Frame@@solution
}
%    \end{macrocode}
%    
%    \begin{macrocode}
\gdef\Puzzle@@normal{
  \let|=\Puzzle@Box@@normal
  \let\Frame=\Puzzle@Frame@@normal
}
\endgroup
%    \end{macrocode}
%    
%    \begin{environment}{Puzzle}
%    The environment |Puzzle| typesets the frame of a crossword
%    puzzle. It is implemented utilizing a |picture| environment. The
%    unilength is set to the |\PuzzleUnitlength|. Thus the navigation
%    is faily easy. The basic unit is width and height of a single
%    box.
%
%    The macros which are local to the environment are activated. Thus
%    we avoid collisions with other packages where the same macro names
%    might be used.
%
%    Finally the counter which contain the x and the y coordinate have
%    to be initialized. 
%
%    The last action in the |picture| environment is the expansion of
%    the macro |\PuzzleHook|. This can be used to include additional
%    material in the |picture| environment. Primarily I have use this
%    to include the ads. But now there is the macro |\Frame| for this
%    purpose. 
%    \begin{macrocode}
\newenvironment{Puzzle}[2]{\par\noindent\mbox{}\hfill
  \catcode`\|=13
  \@nameuse{Puzzle@@\Puzzle@TYPE}%
  \unitlength=\PuzzleUnitlength
  \Puzzle@Y=#2
  \begin{picture}(#1,#2)
    \Puzzle@Box@@normal.
}{%
    \PuzzleHook
  \end{picture}\hfill\null\par\noindent
}
%    \end{macrocode}
%    \end{environment}
%    
%    \begin{macro}{\Puzzle@Frame@@normal}
%    The macro |Puzzle@Frame| is used to place additional
%    rectangular regions into the puzzle frame. This frame can contain
%    arbitrary text which is typeset in a centered environment.
%    
%    This macro takes five arguments. The first two arguments are the
%    coordinates of the upper left corner of the frame. The
%    coordinates are logical coordinates starting from the lower left
%    corner of the puzzle. The next two arguments are the width and the
%    height of the frame given in the number of boxes covered. Finally
%    the fifth argument contains the text which should apear in this
%    frame. 
%    \begin{macrocode}
\newcommand\Puzzle@Frame@@normal[5]{\put(#1,#2){\framebox(#3,#4){%
      \begin{minipage}{#3\unitlength}\begin{center} #5
      \end{center}\end{minipage}}}}
%    \end{macrocode}
%    \end{macro}
%    
%    \begin{macro}{\Puzzle@Frame@@solution}
%    For the solution the framed ads are simply ignored. 
%    \begin{macrocode}
\newcommand\Puzzle@Frame@@solution[5]{}
%    \end{macrocode}
%    \end{macro}
%    
%    \begin{macro}{\PuzzleBlackBox}
%    The macro |\PuzzleBlackBox| is called to typeset the black boxes.
%    It should produce a box of at most width and height of
%    |\PuzzleUnitlength|. 
%    \begin{macrocode}
\newcommand\PuzzleBlackBox{\rule{\PuzzleUnitlength}{\PuzzleUnitlength}}
%    \end{macrocode}
%    \end{macro}
%    
%    \begin{macro}{\Puzzle@Box@@normal}
%    The macro |\Puzzle@Box@@normal| performs all tasks when a box
%    should be typeset in ``normal'' mode. The arguments are evaluated
%    and the appropriate tpye of box typeset or other actions performed.
%    \changes{1.4}{1996/11/25}{Minor bugfix. Using ifx instead of if.}
%    \begin{macrocode}
\newcommand\Puzzle@Box@@normal[2][]{%
  \def\Puzzle@tmp{#2}%
  \if\Puzzle@tmp.
    \Puzzle@X=0\relax \advance\Puzzle@Y-1
  \else
    \ifx\@empty\Puzzle@tmp
    \else
    \if\Puzzle@tmp*
    \put(\Puzzle@X,\Puzzle@Y){\framebox(1,1){\PuzzleBlackBox}}
    \else
    \put(\Puzzle@X,\Puzzle@Y){\framebox(1,1){}}
    \fi
    \fi
    \def\Puzzle@tmp{#1}%
    \ifx\@empty\Puzzle@tmp\else
      \put(\Puzzle@X,\Puzzle@Y){%
        \makebox(1,.95)[tl]{\PuzzleNumberFont\,#1}}%
    \fi
    \advance\Puzzle@X 1
  \fi
}
%    \end{macrocode}
%    \end{macro}
%    
%    \begin{macro}{\Puzzle@Box@@solution}
%    The macro |\Puzzle@Box@@solution|  performs all tasks when a box
%    should be typeset in ``solution'' mode. The arguments are evaluated
%    and the appropriate tpye of box typeset or other actions performed.
%    \changes{1.4}{1996/11/25}{Minor bugfix. Using ifx instead of if.}
%    \begin{macrocode}
\newcommand\Puzzle@Box@@solution[2][]{%
  \def\Puzzle@tmp{#2}%
  \if\Puzzle@tmp.
    \Puzzle@X=0\relax \advance\Puzzle@Y-1
  \else
    \ifx\Puzzle@tmp\@empty
    \else\if\Puzzle@tmp*
    \put(\Puzzle@X,\Puzzle@Y){\framebox(1,1){\PuzzleBlackBox}}
    \else
    \put(\Puzzle@X,\Puzzle@Y){\framebox(1,1){\uppercase{#2}}}%
    \fi
    \fi
    \def\Puzzle@tmp{#1}%
    \ifx\Puzzle@tmp\@empty\else
      \ifPuzzle@SolutionNumbered
      \put(\Puzzle@X,\Puzzle@Y){%
        \makebox(1,.95)[tl]{\PuzzleNumberFont\,#1}}%
    \fi
    \fi
    \advance\Puzzle@X 1
  \fi
}
%    \end{macrocode}
%    \end{macro}
%    
%    \subsection{Clues}
%    
%    \begin{macro}{\Puzzle@Clue@@normal}
%    The first and the third argument are shown as clue. This macro is
%    used for unsolved puzzles.
%    \begin{macrocode}
\newcommand\Puzzle@Clue@@normal[3]{\textsf{#1} #3 }
%    \end{macrocode}
%    \end{macro}
%    
%    \begin{macro}{\Puzzle@Clue@@solution}
%    In solutions clues are simply suppressed. Thus all three
%    arguments are discarted. 
%    \begin{macrocode}
\newcommand\Puzzle@Clue@@solution[3]{}
%    \end{macrocode}
%    \end{macro}
%
%    \begin{environment}{Puzzle@Clues@@normal}
%    The environment |Puzzle@Clues@@normal| is mapped to |PuzzleClues|
%    in ``normal'' mode. It typesets its contents in a minipage of
%    appropriate half textwidth.
%    \begin{macrocode}
\newenvironment{Puzzle@Clues@@normal}[1]{%
  \null\hfill
  \let\Clue\Puzzle@Clue@@normal
  \begin{minipage}[t]{.45\textwidth}%
    \PuzzleClueFont{#1}%
}{\end{minipage}\hfill\null }
%    \end{macrocode}
%    \end{environment}
%    
%    \begin{environment}{Puzzle@Clues@@solution}
%    The environment |Puzzle@Clues@@solution| is mapped to |PuzzleClues|
%    in ``solution'' mode. It just suppressed any output.
%    \begin{macrocode}
\newenvironment{Puzzle@Clues@@solution}[1]{%
  \let\Clue\Puzzle@Clue@@solution
}{}
%    \end{macrocode}
%    \end{environment}
%    
%    \begin{macro}{\PuzzleWordsText}
%    The macro |\PuzzleWordsText| is the text typeset at the beginning
%    of the environment |PuzzleWords|. It takes one argument which is
%    the length of the words listed.
%    \begin{macrocode}
\newcommand\PuzzleWordsText[1]{Words of length #1: }
%    \end{macrocode}
%    \end{macro}
%    
%    \begin{environment}{Puzzle@Words@@normal}
%    The environment |Puzzle@Words@@normal| will be mapped to the
%    environment |PuzzleWords| in ``normal'' mode. It just arranges
%    that words are typeset after the |\PuzzleWordsText| has shown the
%    length of the words. Finally a new paragraph is started.
%    \begin{macrocode}
\newenvironment{Puzzle@Words@@normal}[1]{%
  \PuzzleWordsText{#1}%
  \let\Word\relax
}{\par}
%    \end{macrocode}
%    \end{environment}
%    
%    \begin{environment}{Puzzle@Words@@solution}
%    The environment |Puzzle@Words@@solution| will be mapped to the
%    environment |PuzzleWords| in ``solution'' mode. It arranges
%    things that the contents is silently ignored.
%    \begin{macrocode}
\newenvironment{Puzzle@Words@@solution}[1]{%
  \newcommand\Word[1]{}%
}{}
%    \end{macrocode}
%    \end{environment}
%    
%    \subsection{Numbers}
%    
%    \begin{macro}{\PuzzleNumbers}
%    The macro |\PuzzleNumbers| will produce a list of boxes with
%    numbers for letters. It is intended for numbered crossword puzzles.
%    \begin{macrocode}
\newcommand\PuzzleNumbers[1]{\begingroup
  \@nameuse{Puzzle@@\Puzzle@TYPE}%
  \Puzzle@Y=0
  \Puzzle@X=1
  \unitlength=\PuzzleUnitlength
  \Puzzle@Numbers#1.\endgroup}
%    \end{macrocode}
%    \end{macro}
%    
%    \begin{macro}{\Puzzle@Numbers}
%    The macro |\Puzzle@Numbers| loops through the arguments until it
%    finds a dot. For each argument it produces a box, either with the
%    numbers or with the letters or both, depending on the current
%    settings. 
%
%    The loop is implemented via recursion. The box is tyepset by the
%    \verb/|/ macro which takes care of the current settings. For this
%    purpose this characterhas to be made active temporarily.
%    \begin{macrocode}
\begingroup
\catcode`\|=13
\gdef\Puzzle@Numbers#1{%
  \if#1.
    \let\next\relax
  \else
    \begin{picture}(1,1)
      \xdef\X{\the\Puzzle@X}%
      \Puzzle@X=0
      |[\X]{#1}%
    \end{picture}%
    \let\next\Puzzle@Numbers
    \advance\Puzzle@X 1
  \fi
  \next
}
\endgroup
%    \end{macrocode}
%    \end{macro}
%    
%    \begin{macro}{\PuzzleLettersText}
%    The macro |\PuzzleLettersText| contains the text typeset at the
%    beginning of the |\PuzzleLetters| environment.
%    \begin{macrocode}
\newcommand\PuzzleLettersText{The following letters are used: }
%    \end{macrocode}
%    \end{macro}
%    
%    \begin{macro}{\PuzzleLetters}
%    The macro |\PuzzleLetters| is intended to show the letters used
%    in a numbered crossword puzzle. The argument is the (sorted) list
%    of characters used.
%    \begin{macrocode}
\newcommand\PuzzleLetters[1]{\PuzzleLettersText #1\par}
%    \end{macrocode}
%    \end{macro}
%    
%    \begin{macro}{\Puzzle@TYPE}
%    The macro |\Puzzle@TYPE| contains the type of the puzzle. It is
%    used find the appropriate initialization macro.
%    \begin{macrocode}
\newcommand\Puzzle@TYPE{normal}
%    \end{macrocode}
%    \end{macro}
%    
%    \begin{macro}{\PuzzleSolution}
%    The macro |\PuzzleSolution| arranges everything that the
%    following puzzles are typeset in the ``solution'' mode, i.e. the
%    letters are shown and the clues are suppressed.
%
%    This macros has one optional argument which has to be |true| or
%    |false|. This argument determines whether or not the numbers
%    should also be shown in the solution. The default is |false|
%    which means that the numbers are suppressed in the solution.
%    \begin{macrocode}
\newcommand\PuzzleSolution[1][false]{%
  \@nameuse{Puzzle@SolutionNumbered#1}%
  \let\PuzzleClues\Puzzle@Clues@@solution
  \let\endPuzzleClues\endPuzzle@Clues@@solution
  \let\PuzzleWords\Puzzle@Words@@solution
  \let\endPuzzleWords\endPuzzle@Words@@solution
  \xdef\Puzzle@TYPE{solution}}
%    \end{macrocode}
%    \end{macro}
%    
%    \begin{macro}{\PuzzleUnsolved}
%    The macro |\PuzzleUnsolved| arranges everything that the
%    following puzzles are typeset in the ``unsolved'' mode, i.e. the
%    letters are suppressed and the clues are shown.
%    \begin{macrocode}
\newcommand\PuzzleUnsolved{%
  \let\PuzzleClues\Puzzle@Clues@@normal
  \let\endPuzzleClues\endPuzzle@Clues@@normal
  \let\PuzzleWords\Puzzle@Words@@normal
  \let\endPuzzleWords\endPuzzle@Words@@normal
  \xdef\Puzzle@TYPE{normal}}
%    \end{macrocode}
%    \end{macro}
%
%    The boolean |Puzzle@SolutionNumbered| deternimes whether or
%    not the solution should contain numbers. Initially it is set to
%    ``false''.
%    
%    \begin{macrocode}
\newif\ifPuzzle@SolutionNumbered
\Puzzle@SolutionNumberedfalse
%    \end{macrocode}
%    
%
%    Finally we arrange that the default behaviour is to typeset an
%    unsolved crossword puzzle.
%
%    \begin{macrocode}
\PuzzleUnsolved
%    \end{macrocode}
%
%    That's all.
%
%    \PrintChanges
%    \newpage
%    \PrintIndex
%
%    \Finale
%
\endinput
%
% Local Variables: 
% mode: latex
% TeX-master: t
% End: 
