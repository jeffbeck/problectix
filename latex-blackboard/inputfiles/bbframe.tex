%% Tafelrahmen %%%%%%%%%%%%%%%%%%%%%%%%%%%%%%%%%%%%%%%%%%%%%%%%%%%%%%%%%%%%%%%%%
\begin{picture}(280,24)(0,-4) %% Alle Linien auf dem Blatt
%% Breite Linien zeichnen:
\linethickness{0.6mm}%% Das ist die Liniendicke der dicken Linien
    \put(-0.3,-10){\line(1,0){280.6}}      %% Waagrecht
    \put(-0.3,-115){\line(1,0){280.6}}     %% Waagrecht
    \put(0,-10){\line(0,-1){105}}          %% Senkrecht
    \put(70,-10){\line(0,-1){105}}         %% Senkrecht
    \put(210,-10){\line(0,-1){105}}        %% Senkrecht
    \put(280,-10){\line(0,-1){105}}        %% Senkrecht
%% Schmale Linien Zeichnen
\linethickness{0.3mm}%% Das ist die Liniendicke der d�nnen Linien
    \put(70,-119){\line(0,-1){60}}
    \put(210,-119){\line(0,-1){60}}
%% Text schreiben
    %% Datum
    \put(200,0){\makebox(80,5)[r]{\today}}
    \put(200,-6){\makebox(80,5)[r]{R�diger Beck (Bz)}}
    \put(0,0){\makebox(80,5)[l]{Berufliches Schulzentrum Leonberg}}
    \put(0,-6){\makebox(80,5)[l]{Klasse: \usebox{\klasse}}}
    \bbhead %% \�berschrift Tafelbild Zeichnen
    \put(140,-2){\makebox(0,0)[b]{\underline{Tafelbild}}}
    \bbstep %% Lernschritte-Text
    \linethickness{0.5mm}%%
    %% Berechnung des Mittelpunkts
    \setcounter{stundenthemamitte}{140-\value{stundenthemabreite}/2}
    \put(\value{stundenthemamitte},-16.9)%
        {\framebox(\value{stundenthemabreite},6.6){\usebox{\thema}}}
%% Hilfslinien Schreiben, falls gitter=2
\ifthenelse{\value{gitter}=2}{%
   \linethickness{0.1mm}%% Das ist die Liniendicke der Hilfs- Linien
   \multiput(7,-9.7)(7,0){39}{\put(0,0){\line(0,-1){105.6}}}
   \multiput(0,0)(0,-7){14}{\put(-0.3,-17){\line(1,0){280.6}}}}% Dann
   {}% Sonst nichts tun
%% Hilfskreuze Schreiben, falls gitter=1
\ifthenelse{\value{gitter}=1}{% Beginn Dann
   \linethickness{0.1mm}%% Das ist die Liniendicke der Hilfs- Linien
   \multiput(0,0)(7,0){41}{% aussen
   \multiput(0,0)(0,-7){16}{% innen
   \put(0,-10.5){\line(0,1){1}}\put(-0.5,-10){\line(1,0){1}}
   }% ende multiput innen
   }% ende multiput aussen
   }% Ende Dann
   {}% Sonst nichts tun
\end{picture}








%%% Local Variables: 
%%% mode: latex
%%% TeX-master: t
%%% End: 
