\documentclass{bb}
\bbtopic[105]{Anwendung der Dokumentklasse Tafel}
\Klasse{M2KB}
\setlength{\distanzwert}{1.3mm}% Horizontal zwischen Tafeln

\setlength{\hoehenkorrektur}{-0.3mm}%
\setlength{\grafikhoehenkorrektur}{-4.2mm}
\setlength{\formelhoehenkorrektur}{-4.2mm}

\begin{document}
%%  TAFEL  %%%%%%%%%%%%%%%%%%%%%%%%%%%%%%%%%%%%%%%%%%%%%%%%%%%%%%%%%%%%%%%%%%%%%

\begin{bb}
%%%%%
\begin{bbnarrow}{Einfache Beispiele}
  Einfacher Text wird automatisch umbrochen. \par
  Hier eine Bullet-Aufz�hlung: \par
  \begin{bbitemize}
      \item Der \texttt{erste} Punkt.
      \item Und der Zweite.
  \end{bbitemize}
  Und eine Nummerierte:
  \begin{bbenumerate}
      \item Schnell
      \item Einfach
  \end{bbenumerate}
  F�r Tabellen wird eine angepasste \texttt{tabbing}-Umgebung benutzt:
  \begin{bbtabbing}
      \tab{2} \tab{5} \tab{8} \kill
     \> L�nge \> Breite \> H�he \\
     \> 1mm   \> 7mm    \> 3mm 
  \end{bbtabbing}
  \rule{\linewidth}{1mm}
\end{bbnarrow}
%%%
\begin{bbwide}{Minipages --- Grafik}
  \begin{bbminipage}{11}{4}
    Dies ist eine Text-Minipage. Der Text wird umbrochen, sobald das Ende der
    Minipage erreicht ist. Es folgt eine leere Minipage.
  \end{bbminipage}
  %\grafik{6}{4}{tafelbildtest}
  \bbspace{1}
  \begin{bbminipage}{7}{4}
    Hier eine weitere Text-Minipage. Aber diesmal etwas schm�ler.
  \end{bbminipage}

  \begin{bbminipage}{4}{3}
     Erkl�rung zur Formel
  \end{bbminipage}
  \hspace{7mm}
  \begin{bbformula}{13}{3}
    a=\frac{5+45}{n+2025h^4} \quad mit \quad a=4\,mm^2
  \end{bbformula}
  \hspace{1mm}
 

  \begin{bbminipage}{12}{4}
      Wie das mit Grafiken gemacht wird sieht man auf der n�chsten Seite. Der
      Strich unten gibt die Umbruchsbreite dieser Minipage an. \par
      \rule{\linewidth}{1mm}
  \end{bbminipage}
\end{bbwide}
%%%
\begin{bbnarrow}{Grafiken}
  So sieht eine Grafik aus: \par
  \grafik[4]{8}{8}{bbfile}
  
\begin{bbminipage}{9}{3}
     Hier kommt wieder ein neuer Textblock mit neuem Schwachsinn. 
  \end{bbminipage}

\bbtitle[4]{Zusatzbefehle} \par
Weiter gehts
\end{bbnarrow}
\end{bb}

%%%%%%%%%%%%%%%%%%%%%%%%%%%%%%%%%%%%%%%%%%%%%%%%%%%%%%%%%%%%%%%%%%%%%%%%%%%%%%%%
%%  ANMERKUNGEN %%
%%%%%%%%%%%%%%%%%%%%%%%%%%%%%%%%%%%%%%%%%%%%%%%%%%%%%%%%%%%%%%%%%%%%%%%%%%%%%%%%
\begin{bbnote}
    \begin{bbnarrownote}
        Text
    \end{bbnarrownote}
    \begin{bbwidenote}
        Text der breiten Anmerkung
    \end{bbwidenote}
    \begin{bbnarrownote}
        Text
    \end{bbnarrownote}
\end{bbnote}
\end{document}



%% weiter
%%   bbwide , bbwidenote ersetzten mit wide
%% nummernzaehler ver-englischen














