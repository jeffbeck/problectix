\documentclass{tafel}
\Thema[105]{Anwendung der Dokumentklasse Tafel}
\Klasse{M2KB}
\setlength{\distanzwert}{1.3mm}% Horizontal zwischn Tafeln

\setlength{\hoehenkorrektur}{-0.3mm}%
\setlength{\grafikhoehenkorrektur}{-4.2mm}
\setlength{\formelhoehenkorrektur}{-4.2mm}

\begin{document}
%%  TAFEL  %%%%%%%%%%%%%%%%%%%%%%%%%%%%%%%%%%%%%%%%%%%%%%%%%%%%%%%%%%%%%%%%%%%%%
\begin{tafel}
%%%%%
\begin{schmaletafel}{Einfache Beispiele}
  Einfacher Text wird automatisch umbrochen. \par
  Hier eine Bullet-Aufz�hlung: \par
  \begin{tafelitem}
      \item Der \texttt{erste} Punkt.
      \item Und der Zweite.
  \end{tafelitem}
  Und eine Nummerierte:
  \begin{tafelenum}
      \item Schnell
      \item Einfach
  \end{tafelenum}
  F�r Tabellen wird eine angepasste \texttt{tabbing}-Umgebung benutzt:
  \begin{tafeltabbing}
      \tab{2} \tab{5} \tab{8} \kill
     \> L�nge \> Breite \> H�he \\
     \> 1mm   \> 7mm    \> 3mm 
  \end{tafeltabbing}
  \rule{\linewidth}{1mm}
\end{schmaletafel}
%%%
\begin{breitetafel}{Minipages --- Grafik}
  \begin{tafelminipage}{11}{4}
    Dies ist eine Text-Minipage. Der Text wird umbrochen, sobald das Ende der
    Minipage erreicht ist. Es folgt eine leere Minipage.
  \end{tafelminipage}
  %\grafik{6}{4}{tafelbildtest}
  \leer{1}
  \begin{tafelminipage}{7}{4}
    Hier eine weitere Text-Minipage. Aber diesmal etwas schm�ler.
  \end{tafelminipage}

  \begin{tafelminipage}{4}{3}
     Erkl�rung zur Formel
  \end{tafelminipage}
  \hspace{7mm}
  \begin{formel}{13}{3}
    a=\frac{5+45}{n+2025h^4} \quad mit \quad a=4\,mm^2
  \end{formel}
  \hspace{1mm}
 

  \begin{tafelminipage}{12}{4}
      Wie das mit Grafiken gemacht wird sieht man auf der n�chsten Seite. Der
      Strich unten gibt die Umbruchsbreite dieser Minipage an. \par
      \rule{\linewidth}{1mm}
  \end{tafelminipage}
\end{breitetafel}
%%%
\begin{schmaletafel}{Grafiken}
  So sieht eine Grafik aus: \par
  \grafik[4]{8}{8}{bbfile}
  
\begin{tafelminipage}{9}{3}
     Hier kommt wieder ein neuer Textblock mit neuem Schwachsinn. 
  \end{tafelminipage}

\titel[4]{Zusatzbefehle} \par
Weiter gehts
\end{schmaletafel}
\end{tafel}

%%%%%%%%%%%%%%%%%%%%%%%%%%%%%%%%%%%%%%%%%%%%%%%%%%%%%%%%%%%%%%%%%%%%%%%%%%%%%%%%
%%  ANMERKUNGEN %%
%%%%%%%%%%%%%%%%%%%%%%%%%%%%%%%%%%%%%%%%%%%%%%%%%%%%%%%%%%%%%%%%%%%%%%%%%%%%%%%%
\begin{anmerkung}
    \begin{schmaleanmerkung}
        Text
    \end{schmaleanmerkung}
    \begin{breiteanmerkung}
        Text der breiten Anmerkung
    \end{breiteanmerkung}
    \begin{schmaleanmerkung}
        Text
    \end{schmaleanmerkung}
\end{anmerkung}
\end{document}
















