
\documentclass[fleqn,a4paper,twoside,12pt]{article}
\usepackage{multicol}
\usepackage{lscape}
\usepackage{graphicx}
\usepackage{amssymb}
\usepackage{textcomp}
\usepackage{calc}
\usepackage[T1]{fontenc}
\usepackage[latin1]{inputenc}
\usepackage{ngerman}
\usepackage{fancybox}
\usepackage{marvosym}
\usepackage{fancyhdr}
\usepackage[dvips]{color}
% ams-Befehl square ab jetzt nutzlos, da \square von SIunits neu definiert wird
%\usepackage[cdot,amssymb,thickqspace]{SIunits} 
\usepackage{longtable}
\usepackage{dcolumn}
\usepackage{amsmath}
\usepackage{lastpage}
% Satzspiegel
\setlength{\voffset}{-25.4mm}
\setlength{\hoffset}{-25.4mm}

% uncomment this to the a draft mark accross the page
%\draftcopyFirstPage{1}

\setlength{\textwidth}{190mm}
\setlength{\topmargin}{0mm}
\setlength{\textheight}{270mm}
\setlength{\headheight}{11mm}
\setlength{\headsep}{2mm}
\setlength{\topskip}{0mm}

\setlength{\footskip}{6.5mm}
\setlength{\parindent}{0mm}

% ohne Rand 
\setlength{\oddsidemargin}{9mm}
\setlength{\evensidemargin}{9mm}


\setlength{\fboxsep}{1mm}
\setlength{\fboxrule}{0.35mm}
\setlength{\mathindent}{8mm}

\pagestyle{fancy}

\lhead{}
\chead{}
\rhead{}
\lfoot{Berufliches Schulzentrum Leonberg}


\definecolor{rot}{rgb}{1,0,0} 
\definecolor{gruen}{rgb}{0,0.5,0}
\definecolor{blau}{rgb}{0.2,0,1}

\chead{Pr�fungskorrektur mit \texttt{examplix}}
\rfoot{Created by \texttt{examplix}}
\begin{document}

\section{Allgemeines}

\begin{itemize}
\item Jeder korrigierende Lehrer bekommt ein Verzeichnis mit seinem
  Lehrerkurzzeichen auf Diskette. Darin befinden sich einfache Dateien
  im Format Excel-97
\item Zur handschriftlichen Korrektur drucken sie bitte diese Dateien
  aus.

  Ohne die rote bzw. gr�ne Hintergrundfarbe kann man ein Blatt unter
  Excel (Version 2000) ausdrucken mit:

  Datei -> Seite Einrichten -> Reiter:Tabelle -> Haken bei
  Schwarzweissdruck setzen.

\item Vergewissern sie sich, dass 

  \begin{itemize}
  \item sie alle notwendigen Dateien besitzen, die sie zur Korrektur brauchen.
  \item in diesen Dateien Spalten enthalten sind f�r alle Aufgaben,
    die Sie korrigieren sollen.
  \end{itemize}

  Falls etwas fehlt, informieren sie mich \textbf{sofort}:
  \texttt{bz@bszleo.de}.

\end{itemize}


\section{Anleitung}


\begin{itemize}
\item Der Dateinamen der ausgeteilten Dateien \textbf{MUSS} gleich bleiben!

\item Abgespeichert werden \textbf{MUSS} im Format Excel-97-2000. 

Bei neueren Versionen von M\$-Office (Office 2003, XP, ...) benutzen sie bitte:

Speichern unter -> Dateityp: MS-Excel-97-2000 

\item Tragen sie die Noten an die richtige Stelle ein.
\item \textbf{ALLE} Felder m�ssen ausgef�llt werden.

\item Die vollst�ndig ausgef�llten Dateien speichern sie wieder auf
  der Diskette im Verzeichnis \texttt{result}. 

  Geben sie die Diskette zusammen mit handschriftlichen Korrekturen
  beim Klassenlehrer ab.

  Bitte geben sie nach der Erstkorrektur schon zumindest eine
  Beispieldatei zum testen ab.
  

  \textbf{Achtung:} F�r die Auswertung wird nur die Excel-Tabelle
  herangezogen.  Mit ihrer Unterschrift best�tigen sie, dass in der
  Excel-Datei die korrekten Noten stehen.

\item Verschicken sie aus Datenschutzgr�nden keinesfalls ausgef�llte
  Dateien per Email!
\end{itemize}



\end{document}
