\documentclass{beamer}
\usepackage[T1]{fontenc}
\usepackage[isolatin]{inputenc}
\usepackage{beamerthemesplit}
\usepackage{fancyvrb}
\usepackage{verbatim}

\title{Eine Aufgabendatenbank mit \LaTeX}
\author{R�diger Beck}
\date{6. Februar 2006}

\begin{document}

\begin{frame}
  \titlepage
\end{frame}

%reset to nothing
\author{}
\date{}




\section*{�bersicht}
\begin{frame}
  \tableofcontents
\end{frame}




\section{Tresor und Zugriff}
\title{Tresor und Zugriff}
\begin{frame}
  \titlepage
\end{frame}
\setbeamercolor{normal text}{bg=black}
\begin{frame}
   \includegraphics<1>[height=76.6mm]{cvs-sicher1.jpg}%
   \includegraphics<2>[height=76.6mm]{cvs-sicher2.jpg}%
   \includegraphics<3>[height=76.6mm]{cvs-sicher3.jpg}%
\end{frame}
\setbeamercolor{normal text}{bg=}


\begin{frame}
  \frametitle{Zugriff auf den Tresor (CVS-Server) mit lincvs}
  \begin{overprint}
    \textbf{So geht's:} \par
    \begin{itemize}
      \item<2-> Anlegen des Ordners \texttt{mytex}
      \item<3-> \texttt{lincvs} starten
        \begin{itemize}
        \item<4-> Konfiguration von \texttt{lincvs}
        \item<5-> Profil anlegen
        \item<6-> Auschecken des Moduls \texttt{aufgabendb} nach \texttt{mytex}
        \item<7-> Modul dem Arbeitsbereich hinzuf�gen (\textit{Arbeitskopie})
        \end{itemize}
    \end{itemize}
    \onslide<8->
     \textbf{Fertig!}
  \end{overprint}
\end{frame}






\section{Aufgaben ansehen}
\title{Aufgaben ansehen}
\begin{frame}
  \titlepage
\end{frame}
\setbeamercolor{normal text}{bg=black}
\begin{frame}
   \includegraphics<1>[height=76.6mm]{waiter1.jpg}
\end{frame}
\setbeamercolor{normal text}{bg=}



\subsection*{Vorschau  im Editor \texttt{kile}}
\begin{frame}
  \frametitle{Vorschau im Editor \texttt{kile}}
  \begin{overprint}
    \textbf{So geht's:} \par
    \begin{itemize}
      \item<2-> \texttt{lincvs} starten
      \item<3-> Datei Doppelklicken zum �ffnen mit \texttt{kile}
      \item<4-> Vorschau erzeugen mit \texttt{Shift+F5}
    \end{itemize}
    \onslide<5->
     \textbf{Fertig!}
  \end{overprint}
\end{frame}


\subsection*{Vorschau auf die gesamte Aufgabendatenbank}
\begin{frame}
  \frametitle{Vorschau auf die gesamte Aufgabendatenbank}
  \begin{overprint}
    \textbf{So geht's:} \par
    \begin{itemize}
      \item<2-> Befehl \texttt{problectix --www --tree}
      \item<3-> \ldots Das kann dauern \ldots
      \item<4-> \ldots pro Aufgabe ca. 3 Sekunden \ldots
      \item<5-> \ldots Ende!
      \item<6-> Mit \texttt{mozilla} die Aufgabendatenbank ansehen:

     \textbf{file:///home/lehrer/ab/problectix-tree/html-lehrer/index.html}
    \end{itemize}
    \onslide<7->
     \textbf{Fertig!}
  \end{overprint}
\end{frame}







\section{Klassenarbeit erstellen}
\title{Klassenarbeit erstellen}
\begin{frame}
  \titlepage
\end{frame}
\setbeamercolor{normal text}{bg=black}
\begin{frame}
   \includegraphics<1>[height=76.6mm]{klassenarbeit1.jpg}
\end{frame}
\setbeamercolor{normal text}{bg=}

\begin{frame}
  \frametitle{Klassenarbeit erstellen}
  \begin{overprint}
    \textbf{So geht's:} \par
    \begin{itemize}
      \item<2-> 2 Programme starten:
        \begin{itemize}
        \item<2-> Aufgaben mit \texttt{mozilla} ansehen.
        \item<2-> Editor \texttt{kile} starten.\par
                  Datei \texttt{ka-vorlage.tex} �ffnen. \par
                  \texttt{ka-vorlage.tex} anderem Dateinamen speichern.
        \end{itemize}
      \item<3-> Einf�gebefehl der gew�nschten Aufgabe in
        \texttt{mozilla} ausw�hlen (linke Maustaste).
      \item<4-> Markierter Befehl mit Mittlerer Maustaste in \texttt{kile} einf�gen.
      \item<5-> Klassenarbeit mit \texttt{Shift+F5} anzeigen lassen.
    \end{itemize}
    \onslide<6->
     \textbf{Fertig!}
  \end{overprint}
\end{frame}





\section{Aufgaben erstellen}
\title{Aufgaben erstellen}
\begin{frame}
  \titlepage
\end{frame}
\setbeamercolor{normal text}{bg=black}
\begin{frame}
   \includegraphics<1>[height=76.6mm]{aufgabenerstellen1.jpg}
\end{frame}
\setbeamercolor{normal text}{bg=}


\begin{frame}[fragile]
  \frametitle{Aufgaben erstellen}
  \textbf{Bespielaufgabe:}
\VerbatimInput{aufgabe-001.tex}
\end{frame}



\begin{frame}
  \vfill
  Diese Pr�sentation wurde mit  \LaTeX{} erstellt.  
\end{frame}

\end{document}

%%% Local Variables: 
%%% mode: latex
%%% TeX-master: t
%%% End: 
