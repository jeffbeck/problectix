\documentclass{beamer}
\usepackage[isolatin]{inputenc}
\usepackage{beamerthemesplit}
\usepackage{fancyvrb}

\newcounter{aufgabe}
\setcounter{aufgabe}{1}

\title{Aufgabendatenbank mit \LaTeX}
\author{R\"udiger Beck}

\date{\today}




\begin{document}

\frame{\titlepage}

\section[Outline]{}
\frame{\tableofcontents}

\section{Anforderungen}
\begin{frame}
  \frametitle{Anforderungen}
    \begin{itemize}
      \item<1-> Wiederverwendbarkeit von Aufgaben
      \item<2-> Einfache Auswahl von Aufgaben
      \item<3-> Arbeit an gemeinsamer Datenbank m�glich      
    \end{itemize}
\end{frame}


\section{Einf�hrung in \LaTeX}


\begin{frame}[fragile]
  \frametitle{\LaTeX -Dokument mit 2 Fragen}

\VerbatimInput{code-examples/start-1.tex}


\end{frame}


\frame
{
  \frametitle{Aufgabe \theaufgabe}

 Erstellen sie eine \LaTeX -Datei mit Folgendem Inhalt:

  \begin{itemize}
  \item<1-> Zwei Fragen ohne Verwendung von Formeln.
  \item<-2> Zwei Fragen zu einem anderen Thema mit Formeln
  \end{itemize}

  Benutzen sie eine Leerzeile zur Abtrennung der Aufgaben.

}



\section{Die Dokumentklasse \texttt{teacher}}


\begin{frame}[fragile]
  \frametitle{2 Fragen als Klassenarbeit}

\VerbatimInput{code-examples/teacher-1.tex}


\end{frame}




\subsection{Aufgabe \theaufgabe}
\stepcounter{aufgabe}

\frame
{
  \frametitle{Features of the Beamer Class}

  \begin{itemize}
  \item<1-> Normal LaTeX class.
  \item<2-> Easy overlays.
  \item<3-> No external programs needed.      
  \end{itemize}
}
\frame
{
  \frametitle{}

  \begin{itemize}
  \item<1-> Normal LaTeX class.
  \item<2-> Easy overlays.
  \item<3-> No external programs needed.      
  \end{itemize}
}
\end{document}


