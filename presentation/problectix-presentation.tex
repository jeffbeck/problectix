\documentclass{beamer}
\usepackage[isolatin]{inputenc}
\usepackage{beamerthemesplit}
\usepackage{fancyvrb}
\usepackage{verbatim}

\newcounter{aufgabe}
\setcounter{aufgabe}{1}

\newcounter{beispiel}
\setcounter{beispiel}{1}

\title{Eine Aufgabendatenbank mit \LaTeX}
\author{R�diger Beck}
\date{\today}

\begin{document}

\frame{\titlepage}

%\section[Outline]{}
\section*{\"Ubersicht}
\frame{\tableofcontents}

\section{Anforderungen an ein Aufgabendatenbank}
\begin{frame}
  \frametitle{Anforderungen an ein Aufgabendatenbank}
    \begin{itemize}
      \item<1-> Mehrfache Verwendbarkeit von Aufgaben
        \begin{itemize}
        \item<2-> in Arbeitsbl�ttern
        \item<3-> in Aufgabenbl�ttern
        \item<4-> in Klassenarbeiten
        \item<5-> zur Pr�fungsvorbereitung
        \item<6-> in Pr�fungen
        \end{itemize}
      \item<7-> Einfache Auswahl von Aufgaben (Klassenarbeiten, Pr�fungen)
      \item<8-> Mehrere Kollegen arbeiten an einer gemeinsamen Datenbank      
    \end{itemize}
\end{frame}


\section{Einf�hrung in \LaTeX}

%\section{Beispiel \thebeispiel}


\title{Einf�hrung in \LaTeX}
\author{}
\date{}
\frame{\titlepage}


\begin{frame}[fragile]
  \frametitle{Beispiel \thebeispiel : \LaTeX -Dokument mit 2 Fragen}
  
\VerbatimInput{code-examples/start-1.tex}
\end{frame}
\stepcounter{beispiel}



\begin{frame}
  \frametitle{Befehle in \LaTeX}

   \texttt{\textbackslash documentclass[12pt]\{article\}}

\begin{itemize}
  \item<1-> Befehl \texttt{documentclass}

       $\Longrightarrow$ Vorlage w�hlen
  \item<2-> zwingendes Argument: \texttt{article} 

       $\Longrightarrow$ Vorlage \texttt{article}
  \item<3-> optionales Argument: \texttt{12pt}

       $\Longrightarrow$ Standard-Schriftgr��e \texttt{12pt}
\end{itemize}
\end{frame}




\begin{frame}
  \frametitle{Umgebungen in \LaTeX}

   \texttt{\textbackslash begin\{document\}}

   \texttt{\quad ...}

   \texttt{\textbackslash end\{document\}}

\begin{itemize}
  \item<1-> \texttt{\textbackslash begin\{document\}}
 
       $\Longrightarrow$ Beginn der Umgebung \texttt{document}
  \item<2-> \texttt{...} innerhalb der Umgebung

       $\Longrightarrow$ gelten festgelegte Eigenschaften
  \item<3-> \texttt{\textbackslash end\{document\}}
 
       $\Longrightarrow$ Ende der Umgebung \texttt{document}
\end{itemize}
\end{frame}




\subsection{Aufgabe \theaufgabe}


\begin{frame}
  \frametitle{Aufgabe \theaufgabe:}

 Erstellen sie eine \LaTeX -Datei mit folgendem Inhalt:

  \begin{itemize}
  \item<1-> Zwei Fragen ohne Verwendung von Formeln.
  \item<1-> Zwei Fragen zu einem anderen Thema mit Formeln
  \end{itemize}

  Benutzen sie eine Leerzeile zur Abtrennung der Aufgaben.
\end{frame}
\stepcounter{aufgabe}




\section{Die Dokumentklasse \texttt{teacher}}

\title{Die Dokumentklasse \texttt{teacher}}
\author{}
\date{}
\frame{\titlepage}


\begin{frame}
  \frametitle{Die Dokumentklasse \texttt{teacher}}
  \begin{itemize}
  \item<1-> Laden von Paketen f�r Zusatzfunktionen
    \begin{itemize}
    \item<1-> farbiger Text
    \item<1-> einbinden von Grafiken
    \item<1-> Querformat
    \item<1-> \ldots
    \end{itemize}
  \item<2-> Satzspiegel (Randeinstellungen, Kopf- und Fu�zeile, ...)
  \item<3-> Eigene Befehle
  \item<4-> Eigene Umgebungen (\texttt{aufgabe}, \texttt{teilaufgabe}, ...)
  \end{itemize}

\end{frame}


\begin{frame}[fragile]
  \frametitle{Beispiel \thebeispiel : Fragen als Klassenarbeit}

\VerbatimInput{code-examples/teacher-1.tex}

\end{frame}
\stepcounter{beispiel}




\subsection{Aufgabe \theaufgabe}

\begin{frame}
  \frametitle{Aufgabe \theaufgabe:}

 Stellen sie ihre  \LaTeX -Datei so um, da�

  \begin{itemize}
  \item<1-> die Vorlage \texttt{teacher} benutzt wird:

   $\Longrightarrow$ optionales Argument \texttt{kalsg}
  \item<1-> die Umgebung \texttt{aufgabe} benutzt wird:

   $\Longrightarrow$ \texttt{\textbackslash begin\{aufgabe\}\{Titel\}}
  \item<1-> die Umgebung \texttt{teilaufgabe} benutzt wird:

   $\Longrightarrow$ \texttt{\textbackslash begin\{teilaufgabe\}\{Fach\}\{Linien\}\{Punkte\}}
  \end{itemize}
\end{frame}
\stepcounter{aufgabe}



\begin{frame}[fragile]
  \frametitle{Beispiel \thebeispiel : L�sungen}

\VerbatimInput{code-examples/teacher-2.tex}

\end{frame}
\stepcounter{beispiel}





\subsection{Aufgabe \theaufgabe}

\begin{frame}
  \frametitle{Aufgabe \theaufgabe:}

 Stellen sie ihre  \LaTeX -Datei so um, da�

  \begin{itemize}
  \item<1-> die Umgebung \texttt{loesung} benutzt wird:

   $\Longrightarrow$ nach der Umgebung \texttt{teilaufgabe}
  \item<1-> innerhalb der Umgebung \texttt{loesung} der Befehl \texttt{punkte} 
            benutzt wird:

   $\Longrightarrow$ \texttt{\textbackslash punkte\{L�sung\}\{Punkte\}\{Kommentar\}}
  \end{itemize}
\end{frame}
\stepcounter{aufgabe}



\section{Optionale Argumente der Dokumentklasse}

\begin{frame}
  \frametitle{Optionale Argumente der Dokumentklasse}

Optionale Argumente der Dokumentklasse:

  \begin{itemize}
  \item<1-> Argumente zum Zeigen/Verbergen von Aufgaben-/L�sungsteilen:

   $\Longrightarrow$ \texttt{arb}, \texttt{arblsg}, \texttt{ka}, \texttt{kalsg},
     \texttt{lsg}, \texttt{slsg}, \texttt{\dots}
  \item<1-> Argumente die einen Schulkopf bereitstellen:

   $\Longrightarrow$ \texttt{bszleo}, \texttt{bszleoexam}, 
     \texttt{grundschule}, \texttt{\ldots} 
   \end{itemize}
\end{frame}



\subsection{Aufgabe \theaufgabe}

\begin{frame}
  \frametitle{Aufgabe \theaufgabe:}

  �ndern sie das Aussehen ihrer Datei mit den Argumenten 

  \begin{itemize}
    \item \texttt{arb}, \texttt{arblsg}, \texttt{ka}, \texttt{kalsg},
       \texttt{lsg}, \texttt{slsg}, \texttt{\dots}
    \item \texttt{bszleo}, \texttt{bszleoexam}, 
       \texttt{grundschule}, \texttt{\ldots} 
  \end{itemize}

\end{frame}
\stepcounter{aufgabe}


\section{Aufbau der Datenbank}


\title{Aufbau der Datenbank}
\author{}
\date{}
\frame{\titlepage}


\begin{frame}
  \frametitle{Auslagern von Aufgaben}

  Aufgaben werden ausgelagert, indem

  \begin{itemize}
  \item Eine neue Datei \texttt{aufgabe-xy.tex} angelegt wird, welche die gesamte
    Umgebung \texttt{aufgabe} enth�lt.
    \item In der Klassenarbeit diese Datei mit \texttt{\textbackslash 
       nehme\{datei-xy\} } eingebunden wird.
  \end{itemize}

\end{frame}



\begin{frame}[fragile]
  \frametitle{Beispiel \thebeispiel : Die Klassenarbeitsdatei}
  
\VerbatimInput{code-examples/ka-1.tex}
\end{frame}
\stepcounter{beispiel}

\begin{frame}[fragile]
  \frametitle{Beispiel \thebeispiel : Die Datei \texttt{aufgabe-002.tex}}
  
\VerbatimInput{code-examples/aufgabe-002.tex}
\end{frame}
\stepcounter{beispiel}




\subsection{Aufgabe \theaufgabe}

\begin{frame}
  \frametitle{Aufgabe \theaufgabe:}


  Beginnen Sie mit dem Aufbau der Datenbank

  
  \begin{itemize}
  \item Lagern sie ihre Aufgaben in Dateien mit dem Namen
    \texttt{aufgabe-kuerzel-nummer.tex} aus.

    Die Aufgaben-Dateien m�ssen (vorerst) im selben Verzeichnis wie
    die Klassenarbeits-Datei liegen.
  \item Binden sie die Dateien mit dem Befehl \texttt{\textbackslash
      nehme\{datei\}} wieder ein.
  \end{itemize}

\end{frame}
\stepcounter{aufgabe}


% ausf�llen des Kopfes

% schmankerl

%   \abc
%   Multiple choice
%   \anhang
%   Formeln


\end{document}


