%% this file is bszleoexam-abd.tex
%% It is used AtBeginDocument (abd) when:
%% 'bszleoexam' - Option is given in the package 'teacher'

%%%%%%%%%%%%%%%%%%%%%%%%%%%%%%%%%%%%%%%%%%%%%%%%%%%%%%%%%%%%
%% Beginn: The first page
%%%%%%%%%%%%%%%%%%%%%%%%%%%%%%%%%%%%%%%%%%%%%%%%%%%%%%%%%%%%
\setcounter{teilaufnummerierung}{1}
\setcounter{aufgabennummerierung}{1}
\thispagestyle{empty}
  \renewcommand{\headrulewidth}{0mm}%
\vspace*{-25mm}
\addtocontents{toc}{Diese Seite (DIN\,A3-Umschlag)\dotfill Seite \thepage \par}
\begin{tabular*}{\textwidth}[t]{|p{\textwidth/2-2\tabcolsep}%
                                |p{\textwidth/2-2\tabcolsep}|} \hline 
 & \\[-4mm] %% necessary
\multicolumn{1}{|l}{Abschlussprüfung}  & 
\multicolumn{1}{|r|}{Abschlussprüfung} \\
\multicolumn{1}{|l}{der Berufsschulen (gewerbl. Bereich)} & 
\multicolumn{1}{|r|}{der Handwerkskammern} \\
\multicolumn{1}{|l}{Ministerium für Kultus und Sport} & 
\multicolumn{1}{|r|}{ (schriftlicher Teil)} \\
\multicolumn{1}{|l}{Baden-Württemberg} & 
\multicolumn{1}{|r|}{ Baden-Württemberg} \\ \hline
\end{tabular*}

\begin{center}
   \bigskip  
   \bigskip  
   {\huge \bf \datumuse{}}

   \ifthenelse{\value{projektbezug}=0}{\bigskip}{\vspace{2mm}}
  
   \huge \textbf{Kälteanlagenbauer}

   \ifthenelse{\value{projektbezug}=0}{\bigskip}{\vspace{2mm}}
   \bigskip  
\end{center}

\begin{center}
\setlength{\fboxrule}{0.3mm}
\setlength{\fboxsep}{2mm}
\framebox[\textwidth][s]   
%(projektunabhängig) %
\hspace*{\fill} Bearbeitungszeit: \totaltimeuse{} Minuten}
\end{center}

\bigskip  

\begin{center}
\begin{tabular}[t]{|p{\textwidth/5}|c|c|c|} \hline
\bfseries \rule[-3mm]{0mm}{9mm}Fach: & 
Arbeitsplanung &  Technologie & Technische Mathematik \\ \hline
\bfseries\rule[-3mm]{0mm}{9mm}Richtzeit: &  \aptimeuse{} Minuten &  
\ttimeuse{} Minuten & \mtimeuse{} Minuten\\ \hline
\end{tabular}

\ifthenelse{\value{projektbezug}=0}{\bigskip}{\vspace{2mm}}
\bigskip  

\end{center}

\begin{tabbing}
\hspace*{43mm}\= neu\= \kill
\textbf{Verlangt:} \> Es sind alle Aufgaben zu lösen. \\[7mm]
\textbf{Hilfsmittel:} 
    \> Tabellenbuch, Formelsammlung, Taschenrechner, Zeichengeräte. \\[7mm]
 \textbf{Bewertung:} 
    \> Die Aufgaben eines Faches aus den Prüfungsblöcken Fachtheorie I \\
    \>und Fachtheorie II werden zusammen gewertet.\\[7mm]
\textbf{Zu beachten:} 
    \> Die Prüfungsunterlagen bestehen aus \pageref{VeryLastPage} Seiten. \\
    \> Prüfen Sie am Beginn der Prüfung die Vollständigkeit. \\[7mm]
    \> Bei Unstimmigkeiten ist sofort die Prüfungsaufsicht zu Informieren. \\[7mm]
\end{tabbing}

\vspace{-12mm}

\hspace*{42mm}
\begin{minipage}[t]{110mm}
   \tableofcontents
\end{minipage}

\bigskip

\begin{tabular*}{\textwidth}[t]{|p{\textwidth/5}|@{\extracolsep{\fill}} l|} \hline
\bfseries\rule[-3mm]{0mm}{9mm}Name, Vorname: &  \\ \hline
\bfseries\rule[-3mm]{0mm}{9mm}Klasse:        &  \\ \hline
\bfseries\rule[-3mm]{0mm}{9mm}Klassenlehrer: &  \\ \hline
\end{tabular*}

\enlargethispage{15mm}

\aufgabenmarkesetzen%

%%\afterpage{% Ab der zweiten Seite wird Rahmen geändert >
\lhead{%
    %% picture as a header
    \begin{picture}(0,20)(0,2)   
          \pruefungspic
    \end{picture}
%%}% 
\chead{\raisebox{15mm}{\thepage}}
\rhead{}% of the first page 
%% 
\lfoot{\revisionuse{}}% of the first page 
%%\cfoot{Seite \thepage /\pageref{VeryLastPage}}% of the first page
\cfoot{}
\rfoot{\ausdruckuse{}}% of the first page
%%
%%
%%%%%%%%%%%%%%%%%%%%%%%%%%%%%%%%%%%%%%%%%%%%%%%%%%%%%%%%%%%%
%% End: From Page 2 on
%%%%%%%%%%%%%%%%%%%%%%%%%%%%%%%%%%%%%%%%%%%%%%%%%%%%%%%%%%%%
     }%
%%
\pagebreak 
%%% Local Variables: 
%%% mode: latex
%%% TeX-master: t
%%% TeX-master: t
%%% End: 
