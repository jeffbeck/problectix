%%%%%%%%%%%%%%%%%%%%%%%%%%%%%%%%%%%%%%%%%%%%%%%%%%%%%%%%%%%%%%%%%%%%%%%%%%%%%%%%
%% Befehl, der den Arbeitsblatt-Kopf l�dt
%%%%%%%%%%%%%%%%%%%%%%%%%%%%%%%%%%%%%%%%%%%%%%%%%%%%%%%%%%%%%%%%%%%%%%%%%%%%%%%%
\newcommand{\arbeitsblattpic}{
%% Breite Linien zeichnen:
  \linethickness{0.5mm}%% Das ist die Liniendicke der breiten Linien
  %% Waagrecht
  \put(-2,20){\line(1,0){180}}            
  \put(-2,0){\line(1,0){180}} %% doppeltvorhanden
  %% Senkrecht
  \put(-2,20.25){\line(0,-1){20.5}}      \put(178,20.25){\line(0,-1){20.5}}
%% Schmale Linien im Kopf zeichnen:
  \linethickness{0.25mm}%% Das ist die Linienbreite der schmalen Linien
  %% Waagrecht 
  \put(30,5){\line(1,0){87}} 
  % Senkrecht
  \put(30,0){\line(0,1){20}}              \put(117,0){\line(0,1){20}}
%% Text
  %% kleinen Text in Blattkopf zeichnen (auch bei L�sungen): 
  \scriptsize
  \put(31.2,1.8){\usertokenuse{}: \fachuse{}}
%  \put(0,-256.8){\makebox(176,4)[r]{{\usebox{\quelle}}}} %% Quellenangabe erm�glichen
  \put(0,-256.8){\makebox(176,4)[r]{\quelleuse{}}} %% Quellenangabe erm�glichen
  \put(56,0.54){\makebox(60,4)[r]{{\jobname.tex(\thepage):\RCSMaster}}}
  %% gro�en Text (Fett) in Blattkopf zeichnen:
  \normalsize \bf
  \put(0,14){Berufliches}   \put(0,8.5){Schulzentrum}   \put(0,3){Leonberg}
  \put(73.5,13.8){\makebox(0,0)[b]{{\titelouse{}}}}
  \put(73.5,8){\makebox(0,0)[b]{\titeluuse{}}}
}
%%%%%%%%%%%%%%%%%%%%%%%%%%%%%%%%%%%%%%%%%%%%%%%%%%%%%%%%%%%%%%%%%%%%%%%%%%%%%%%%
%% Befehl, der den Arbeitsblatt-Kopf l�dt
%%%%%%%%%%%%%%%%%%%%%%%%%%%%%%%%%%%%%%%%%%%%%%%%%%%%%%%%%%%%%%%%%%%%%%%%%%%%%%%%


%%%%%%%%%%%%%%%%%%%%%%%%%%%%%%%%%%%%%%%%%%%%%%%%%%%%%%%%%%%%%%%%%%%%%%%%%%%%%%%%
%% Befehl, der die Namensbereich-Unterteilung l�dt
%%%%%%%%%%%%%%%%%%%%%%%%%%%%%%%%%%%%%%%%%%%%%%%%%%%%%%%%%%%%%%%%%%%%%%%%%%%%%%%%
\newcommand{\namensbereichpic}{
  \linethickness{0.25mm}%% Das ist die Linienbreite der schmalen Linien
  \put(117,12){\line(1,0){61}}   \put(165,0){\line(0,1){12}}
  \put(147,20){\line(0,-1){8}} 
  \put(133,3.3){\LARGE \textcolor{arblsgcolor}{L�sung}} 
  \scriptsize \rm %% \rm gebraucht um von Fett zur�ckzuschalten
  \put(118.2,8.8){Name: }        \put(118.2,16.5){Datum: \datumuse{}}
  \put(166.2,8.8){Blatt:}        \put(148.2,16.5){Klasse: \klasseuse{}}
  \normalsize %% Eintragungen in Blattkopf zeichnen:
  \put(120.2,3){\nameuse{}}  \put(168.2,3){\blattuse{}}
}
%%%%%%%%%%%%%%%%%%%%%%%%%%%%%%%%%%%%%%%%%%%%%%%%%%%%%%%%%%%%%%%%%%%%%%%%%%%%%%%%
%% Befehl, der die Namensbereich-Unterteilung l�dt
%%%%%%%%%%%%%%%%%%%%%%%%%%%%%%%%%%%%%%%%%%%%%%%%%%%%%%%%%%%%%%%%%%%%%%%%%%%%%%%%



%%%%%%%%%%%%%%%%%%%%%%%%%%%%%%%%%%%%%%%%%%%%%%%%%%%%%%%%%%%%%%%%%%%%%%%%%%%%%%%%
%% Befehl, der die Namensbereich-Unterteilung l�dt
%%%%%%%%%%%%%%%%%%%%%%%%%%%%%%%%%%%%%%%%%%%%%%%%%%%%%%%%%%%%%%%%%%%%%%%%%%%%%%%%




%%%%%%%%%%%%%%%%%%%%%%%%%%%%%%%%%%%%%%%%%%%%%%%%%%%%%%%%%%%%%%%%%%%%%%%%%%%%%%%%
%% Befehl, der den Rahmen um das ganze Blatt l�dt
%%%%%%%%%%%%%%%%%%%%%%%%%%%%%%%%%%%%%%%%%%%%%%%%%%%%%%%%%%%%%%%%%%%%%%%%%%%%%%%%
\newcommand{\arbeitsblattrahmenpic}{
\linethickness{0.5mm}%% Das ist die Linienbreite der breiten Linien
%% Waagrecht
\put(-2,20){\line(1,0){180}}   %% oben, doppelt vorhanden
\put(-2,-257){\line(1,0){180}} %% unten
%% Senkrecht
\put(-2,20.25){\line(0,-1){277.5}}    
\put(178,20.25){\line(0,-1){277.5}}
}
\newcommand{\arbeitsblattrahmenpicafive}{
\linethickness{0.5mm}%% Das ist die Linienbreite der breiten Linien
%% Waagrecht
\put(-2,20){\line(1,0){180}}   %% oben, doppelt vorhanden
\put(-2,-114){\line(1,0){180}} %% unten
%% Senkrecht
\put(-2,20.25){\line(0,-1){134.5}}    
\put(178,20.25){\line(0,-1){134.5}}
}
%%%%%%%%%%%%%%%%%%%%%%%%%%%%%%%%%%%%%%%%%%%%%%%%%%%%%%%%%%%%%%%%%%%%%%%%%%%%%%%%
%% Befehl, der den Rahmen um das ganze Blatt l�dt
%%%%%%%%%%%%%%%%%%%%%%%%%%%%%%%%%%%%%%%%%%%%%%%%%%%%%%%%%%%%%%%%%%%%%%%%%%%%%%%%


%%%%%%%%%%%%%%%%%%%%%%%%%%%%%%%%%%%%%%%%%%%%%%%%%%%%%%%%%%%%%%%%%%%%%%%%%%%%%%%%
%% Befehl, der den Rahmen um das ganze Blatt l�dt (Seite2)
%%%%%%%%%%%%%%%%%%%%%%%%%%%%%%%%%%%%%%%%%%%%%%%%%%%%%%%%%%%%%%%%%%%%%%%%%%%%%%%%
\newcommand{\arbeitsblattrahmenpicsec}{
\linethickness{0.5mm}%% Das ist die Linienbreite der breiten Linien
\put(0,4){%% verschiebt den ganzen Rahmen
%% Waagrecht
\put(-2,3){\line(1,0){180}}   %% oben, doppelt vorhanden
\put(-2,-274){\line(1,0){180}} %% unten
%% Senkrecht
\put(-2,3.25){\line(0,-1){277.5}}    
\put(178,3.25){\line(0,-1){277.5}}
}
}
%%%%%%%%%%%%%%%%%%%%%%%%%%%%%%%%%%%%%%%%%%%%%%%%%%%%%%%%%%%%%%%%%%%%%%%%%%%%%%%%
%% Befehl, der den Rahmen um das ganze Blattl�dt
%%%%%%%%%%%%%%%%%%%%%%%%%%%%%%%%%%%%%%%%%%%%%%%%%%%%%%%%%%%%%%%%%%%%%%%%%%%%%%%%


%%%%%%%%%%%%%%%%%%%%%%%%%%%%%%%%%%%%%%%%%%%%%%%%%%%%%%%%%%%%%%%%%%%%%%%%%%%%%%%%
%% Befehl, der den Klassenarbeit-Zusatzrahmen l�dt
%%%%%%%%%%%%%%%%%%%%%%%%%%%%%%%%%%%%%%%%%%%%%%%%%%%%%%%%%%%%%%%%%%%%%%%%%%%%%%%%
\newcommand{\kazusatzpic}{
%% Breite Linien zeichnen >
  \linethickness{0.5mm}%% Das ist die Linienbreite der breiten Linien
  %% Waagrecht
  \put(-2,-12){\line(1,0){180}}
  \put(-2,0){\line(1,0){180}} %% doppelt vorhanden 
  %% Senkrecht
  \put(41,0){\line(0,-1){12}}
  \put(88,0){\line(0,-1){12}}
  \put(135,0){\line(0,-1){12}}
  \put(-2,0.25){\line(0,-1){12.5}}
  \put(178,0.25){\line(0,-1){12.5}}
%% Text 
\normalsize \bf
 \put(138,-7.5){Note\,:}
 \put(44,-7.5){\usebox{\muendlich}}
 \put(91,-7.5){Klassen\,-\,{\O}\,:}
 \put(1,-7.5){\usebox{\fehler}}
}
%%%%%%%%%%%%%%%%%%%%%%%%%%%%%%%%%%%%%%%%%%%%%%%%%%%%%%%%%%%%%%%%%%%%%%%%%%%%%%%%
%% Befehl, der den Klassenarbeit-Zusatzrahmen l�dt
%%%%%%%%%%%%%%%%%%%%%%%%%%%%%%%%%%%%%%%%%%%%%%%%%%%%%%%%%%%%%%%%%%%%%%%%%%%%%%%%



%%%%%%%%%%%%%%%%%%%%%%%%%%%%%%%%%%%%%%%%%%%%%%%%%%%%%%%%%%%%%%%%%%%%%%%%%%%%%%%%
%% Befehl, der den Verbesserungs-Zusatzrahmen l�dt
%%%%%%%%%%%%%%%%%%%%%%%%%%%%%%%%%%%%%%%%%%%%%%%%%%%%%%%%%%%%%%%%%%%%%%%%%%%%%%%%
\newcommand{\kaverbesserungpic}{
%% Breite Linien zeichnen >
   \linethickness{0.5mm} 
   %% Waagrecht
   \put(-2,-12){\line(1,0){180}}
   \put(-2,0){\line(1,0){180}} %% doppelt vorhanden
   %% Senkrecht
   \put(-2,0.25){\line(0,-1){12.5}}
   \put(178,0.25){\line(0,-1){12.5}}
%% Schmale Linien zeichnen >
   \linethickness{0.1mm}
   \put(-2,-1){\line(1,0){180}}   \put(-2,-11){\line(1,0){180}}
%% Text 
\normalsize \bf %% 
 %%KA-Zusatz:
 \put(88,-8){\makebox(0,0)[b]{--- \,Verbesserung\,
 --- \qquad --- \,Verbesserung\,
 --- \qquad --- \,Verbesserung\,
 ---
 }}}
%%%%%%%%%%%%%%%%%%%%%%%%%%%%%%%%%%%%%%%%%%%%%%%%%%%%%%%%%%%%%%%%%%%%%%%%%%%%%%%% 
%% Befehl, der den Verbesserungs-Zusatzrahmen l�dt
%%%%%%%%%%%%%%%%%%%%%%%%%%%%%%%%%%%%%%%%%%%%%%%%%%%%%%%%%%%%%%%%%%%%%%%%%%%%%%%%


\newcommand{\pruefungspic}{
%% Breite Linien zeichnen:
  \linethickness{0.35mm}%% Das ist die Liniendicke der breiten Linien
  %% Waagrecht
  \put(111,2){\line(1,0){52}}            
%% Text
  \normalsize \bf
  \put(0,14){Abschlusspr�fung \datumuse{}}
  \put(97,3){Name: }
  \put(0,8.5){K�lteanlagenbauer}
\ifthenelse{\value{projektbezug}=0}{
\put(0,3){Fachtheorie \usebox{\examtypenumber} (projektunabh�ngig)}
}{
\put(0,3){Fachtheorie \usebox{\examtypenumber} (projektbezogen)}
}   
}

%%% Local Variables: 
%%% mode: plain-tex
%%% TeX-master: t
%%% End: 
