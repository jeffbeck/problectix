\begin{picture}(0,20)(0,2)
%% Breite Linien zeichnen:
\linethickness{0.5mm}%% Das ist die Liniendicke der dicken Linien
 \put(-2,0){\line(1,0){180}}
 \put(-2,20){\line(1,0){180}}
 \put(-2,20.25){\line(0,-1){20.5}}
 \put(178,20.25){\line(0,-1){20.5}}
%% Text %% Text %% Text %% Text %% Text %% Text %% Text %% Text %% Text %% Text %% Text
 \put(0,12){\makebox(176,6)[c]{{\usebox{\titelo}}}}
 \put(0,5.5){\makebox(176,6)[c]{{\usebox{\titelu}}}}
\scriptsize
 \ifthenelse{\boolean{datum}}{% 
   \put(0,-1){\makebox(175,6.5)[r]{\usebox{\datum}}}% Datum anzeigen
   }{
   % Ausdruckdatum anzeigen
   \put(0,-1){\makebox(175,6.5)[r]{Ausdruck: \today\, (\usebox{\usertoken})}}
   }
 \put(0,-1){\makebox(177.5,6.5)[l]{\usebox{\quelle}}}
 \put(0,-1){\makebox(176,6.5)[c]{{\jobname.tex}}}
%%\normalsize
\end{picture}

%%% Local Variables: 
%%% mode: latex
%%% TeX-master: t
%%% End: 
