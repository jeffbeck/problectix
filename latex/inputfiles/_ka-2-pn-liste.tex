%%%%%%%%%%%%%%%%%%%%%%%%%%%%%%%%%%%%%%%%%%%%%%%%%%%%%%%%%%%%%%%%%%%%%%%%%%%%%%%%
%% Punkte-Notenliste einfuegen:
%\newcounter{pnpunkte}

%\newcounter{notenbox}% Anzahl der Notenboxen
%\newcounter{notenboxort}% Wo steht die Notenbox
%\newcounter{schritt}% Breite der Notenboxen
%\newcounter{gesamtpunktzahlplus}
%\newcounter{gesamtpunktzahlplusplus}
%\newcounter{textfeldhoehe}
\setcounter{textfeldhoehe}{4}
\setcounter{pnpunkte}{4}

%\newcounter{doppeltextfeldhoehe}
\setcounter{doppeltextfeldhoehe}{\value{textfeldhoehe}} 
\addtocounter{doppeltextfeldhoehe}{\value{textfeldhoehe}}

\setcounter{gesamtpunktzahlplus}{\value{gesamtpunktzahl}}
\setcounter{gesamtpunktzahlplusplus}{\value{gesamtpunktzahl}}
\addtocounter{gesamtpunktzahlplus}{1}
\addtocounter{gesamtpunktzahlplusplus}{2}
\setcounter{schritt}{178/\real{\value{gesamtpunktzahlplus}}}
\setcounter{notenboxort}{178} % Erste Notenbox ganz rechts aussen
\addtocounter{notenboxort}{-\value{schritt}}
\addtocounter{gesamtpunktzahlplus}{1}
\linethickness{0.2mm}%% Linienbreite der Zwischenlinien
%\whiledo{\value{notenbox}<\value{gesamtpunktzahlplus}}% Test
\whiledo{\value{notenbox}<7}% Test
{% Schleifentext Anfang
   \put(\value{notenboxort},9){\line(0,-1){\value{doppeltextfeldhoehe}}}
   \put(\value{notenboxort},9){\makebox(\value{schritt},-2){\scriptsize \thepnpunkte}}
    \put(\value{notenboxort},0){\makebox(\value{schritt},\value{textfeldhoehe}){\scriptsize y}}
    \addtocounter{notenboxort}{-\value{schritt}}
    \addtocounter{notenbox}{1}
    \addtocounter{pnpunkte}{1}
}% Schleifentext Ende
\put(-1,0){\makebox(8,\value{textfeldhoehe})[l]{\scriptsize Punkte}}
\put(-1,5){\makebox(8,\value{textfeldhoehe})[l]{\scriptsize Note}}

\linethickness{0.2mm}%% Linienbreite der Zwischenlinien
\put(-2.25,9){\line(1,0){180}}
\put(-2.25,5){\line(1,0){180}}

\linethickness{0.5mm}%% Linienbreite der Aussenlinien
\put(178,8){\line(0,-1){\value{doppeltextfeldhoehe}}}
\put(-2,8){\line(0,-1){\value{doppeltextfeldhoehe}}}
\put(-2.25,0){\line(1,0){180.5}}


%% Weiter:
%% schrittweite nicht als integer, sondern genau berechnen lassen wie?
%% fuer punkte und noten soviel platz lassen, wie gebraucht wird
%% notenberechnung durchfuehren und damit y ersetzen
%% eingabe fuer bei wieviel punkten gibt es eine 6 vorsehen
%% befehl im vorspann, der dieses notenliste mit entsprechenden parametern aufruft

%%% Local Variables: 
%%% mode: plain-tex
%%% TeX-master: t
%%% End: 
