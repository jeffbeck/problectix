\typeout{Loading all-format.tex}
\setcounter{loesungslinienzeigen}{1}
\setcounter{teilaufnummerierung}{0}
\setcounter{aufgabenpunkte}{0}
\setcounter{aufgabenkopfzeile}{1}
\setcounter{aufgabennummerierung}{0}
\setcounter{aufgabenstellung}{1}
\setcounter{aufgabenfusszeile}{1} %
\setcounter{loesungkopfzeile}{0}
\setcounter{loesungen}{0}
\setcounter{xlsg}{1}            
\setcounter{dateinamen}{1}        
\setcounter{dehnen}{0}    
\setcounter{fachangabe}{1}        
\setcounter{fachangabepzk}{1}
\setcounter{punkteangabepzk}{1}
\setcounter{punktzahlkasten}{1}
\setcounter{punktesummezeigen}{1}  %
\setcounter{aufgabentitel}{1}     
\setcounter{gruppeninfo}{2}    
\setcounter{gruppeninfohead}{0}    %
\setcounter{notenliste}{1}
\setlength{\lkaabstand}{3mm}
\setlength{\zusatzlkaabstand}{1mm}

%% Farbe auf gr�n setzten
\definecolor{xlsgcolor}{rgb}{0,0.5,0}
\definecolor{xlsgmustercolor}{rgb}{0.6,0.1,0.7}

\endinput


%%% Local Variables: 
%%% mode: plain-tex
%%% TeX-master: t
%%% End: 
