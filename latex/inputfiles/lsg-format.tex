%^^A Diese Datei wird bei der Angabe der Option 'ka' in der 
%^^A Dokumentklasse 'teacher' eingelesen.
%^^A Es dient dazu die Formatierung der erzeugten Klassenarbeit zu steuern

\typeout{Loading lsg-format.tex}

% \DescribeMacro{teilaufnummerierung} Art der Nummerierung der 
% Aufgaben: a,b,c... (=0) oder 1.1,1.2,1.3...(=1)
\setcounter{teilaufnummerierung}{0}

% \DescribeMacro{aufgabenpunkte} Punktzahl zeigen = (1), verbergen = (0). 
% Gemeint ist die Punktzahl am Ende einer ganzen  Aufgabe.
\setcounter{aufgabenpunkte}{0}

% \DescribeMacro{aufgabenkopfzeile}
% Schaltet die Ausgabe der Aufgaben-Kopfzeile ein = (1), aus = (0).
\setcounter{aufgabenkopfzeile}{0}

% \DescribeMacro{aufgabenstellung}
% Aufgabenstellung (Fragen) zeigen = (1), verbergen = (0).
\setcounter{aufgabenstellung}{0}

% \DescribeMacro{aufgabenfusszeile}
% Schaltet die Ausgabe der Aufgaben-Fusszeile ein = (1), aus = (0).
\setcounter{aufgabenfusszeile}{0}

% \DescribeMacro{loesungkopfzeile}
% Schaltet die Ausgabe der L�sungs-Kopfzeile ein = (1), aus = (0).
\setcounter{loesungkopfzeile}{1}

% \DescribeMacro{loesungen}
% L�sungen (Antworten) zu den Aufgaben zeigen = (1), verbergen = (0).
\setcounter{loesungen}{1}

\setcounter{xlsg}{0}

% \DescribeMacro{dateinamen}
% Dateinamen in der Aufgaben-Kopfzeile zeigen = (1), verbergen = (0).
\setcounter{dateinamen}{1} 

% \DescribeMacro{dehnen}
% Variable Zwischenabst�nde ja = (1), nein = (0) (nicht vollst�ndig implementiert)
\setcounter{dehnen}{0}

% \DescribeMacro{fachangabe}
% Fach der Aufgabe im Aufgabentitel-Kopf zeigen = (1), oder aus = (0)
\setcounter{fachangabe}{1}

% \DescribeMacro{fachangabepzk}
% Fachangabe-Buchstaben (M, T, AP) am Punktzahl-Kasten 
% zeigen = (1), oder aus = (0)
\setcounter{fachangabepzk}{0}

% \DescribeMacro{punkteangabepzk}
% Erreichbare Punktezahl der Teilaufgabe am Punktzahl-Kasten 
% zeigen = (1), oder aus = (0)
\setcounter{punkteangabepzk}{0}

% \DescribeMacro{punktzahlkasten}
% Punktzahl-Kasten zeigen = (1), oder aus = (0)
\setcounter{punktzahlkasten}{0}

% \DescribeMacro{punktesummezeigen}
% Punktesumme der Teilaufgaben am Ende der Aufgaben
% zeigen = (1), oder aus = (0). Veraltet.
\setcounter{punktesummezeigen}{0}

% \DescribeMacro{aufgabentitel}
% Name der Aufgabe hinter der Aufgabenmarke im Aufgabentitel-Kopf
% zeigen = (1), oder aus = (0)
\setcounter{aufgabentitel}{0}

% \DescribeMacro{gruppeninfo}
% Informationen �ber die Gruppen in der Aufgaben-Fusszeile zeigen.
% aus = (0), SW = (1) oder in Farbe = (1) ?????
% Farbe/SW wird mit bw
\setcounter{gruppeninfohead}{1}
\setcounter{gruppeninfo}{0}

% \DescribeMacro{lkaabstand}
% L�ngenangabe.
% Zusatzabstand der L�sungslinien f�r den lka-Befehl.
% kommt zum Zeilenumbruch noch hinzu???
\setlength{\lkaabstand}{3mm}

% \DescribeMacro{zusatzlkaabstand}
% L�ngenangabe.
% Zus�tzlich zur L�nge lkaabstand vor der ersten L�sungslinie eingef�gt, 
% um Aufgabe und L�sung klarer zu trennen.
\setlength{\zusatzlkaabstand}{1mm}

% \DescribeMacro{debuggen}
% Schaltet Hilfen zur Aufgabenerstellung ein = (1), oder aus = (0).
% Z. B. Lineale, etc \ldots
\setcounter{debuggen}{0}

% \DescribeMacro{loesungslinienzeigen}
% Zeigt L�sungslinien oder verbirgt sie.
\setcounter{loesungslinienzeigen}{0}


\setcounter{notenliste}{0}%%

%%% Local Variables: 
%%% mode: plain-tex
%%% TeX-master: t
%%% End: 
