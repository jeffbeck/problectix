%% Befehle, die bei jedem Fach-Picture gleich bleiben
\newcommand{\kaendefeld}{%
\linethickness{0.5mm}%% Linienbreite
 \put(-2,0){\line(1,0){180}}
 \put(-2,9){\line(1,0){180}}
 \put(-2,-0.25){\line(0,1){9.5}}
 \put(34,-0.25){\line(0,1){9.5}}
 \put(70,-0.25){\line(0,1){9.5}}
 \put(95,-0.25){\line(0,1){9.5}}
 \put(126,-0.25){\line(0,1){9.5}}
 \put(164,-0.25){\line(0,1){9.5}}
 \put(178,-0.25){\line(0,1){9.5}}
%% TEXT %% TEXT %% TEXT %% TEXT %%%%%%%%%%%%%%%%%%%%%%%%%%%%%%%%%%%%%%%%%%%%%%%%
\normalsize\bf %% 
 \put(36,3){M�ndl.:}
 \put(72,3){{\O}\,:}
\put(97,3){Note:}
%%%%%%%%%%%%%%%%%%%%%%%%%%%%%%%%%%%%%%%%%%%%%%%%%%%%%%%%%%%%%%%%%%%%%%%%%%%%%%%%%
%% Punkte-Notenliste einfuegen:
%\newcounter{pnpunkte}

%\newcounter{notenbox}% Anzahl der Notenboxen
%\newcounter{notenboxort}% Wo steht die Notenbox
%\newcounter{schritt}% Breite der Notenboxen
%\newcounter{gesamtpunktzahlplus}
%\newcounter{gesamtpunktzahlplusplus}
%\newcounter{textfeldhoehe}
\setcounter{textfeldhoehe}{4}
\setcounter{pnpunkte}{4}

%\newcounter{doppeltextfeldhoehe}
\setcounter{doppeltextfeldhoehe}{\value{textfeldhoehe}} 
\addtocounter{doppeltextfeldhoehe}{\value{textfeldhoehe}}

\setcounter{gesamtpunktzahlplus}{\value{gesamtpunktzahl}}
\setcounter{gesamtpunktzahlplusplus}{\value{gesamtpunktzahl}}
\addtocounter{gesamtpunktzahlplus}{1}
\addtocounter{gesamtpunktzahlplusplus}{2}
\setcounter{schritt}{178/\real{\value{gesamtpunktzahlplus}}}
\setcounter{notenboxort}{178} % Erste Notenbox ganz rechts aussen
\addtocounter{notenboxort}{-\value{schritt}}
\addtocounter{gesamtpunktzahlplus}{1}
\linethickness{0.2mm}%% Linienbreite der Zwischenlinien
\whiledo{\value{notenbox}<\value{gesamtpunktzahlplus}}% Test
{% Schleifentext Anfang
\put(\value{notenboxort},0){\line(0,-1){\value{doppeltextfeldhoehe}}}
\put(\value{notenboxort},-\value{textfeldhoehe}){\makebox(\value{schritt},\value{textfeldhoehe}){\scriptsize \thepnpunkte}}
\put(\value{notenboxort},-\value{doppeltextfeldhoehe}){\makebox(\value{schritt},\value{textfeldhoehe}){\scriptsize y}}
\addtocounter{notenboxort}{-\value{schritt}}
\addtocounter{notenbox}{1}
\addtocounter{pnpunkte}{1}
}% Schleifentext Ende
\put(-1,-\value{textfeldhoehe}){\makebox(8,\value{textfeldhoehe})[l]{\scriptsize
    Punkte}}
\put(-1,-\value{doppeltextfeldhoehe}){\makebox(8,\value{textfeldhoehe})[l]{\scriptsize Note}}
\linethickness{0.2mm}%% Linienbreite der Zwischenlinien
\put(-2.25,-\value{textfeldhoehe}){\line(1,0){180}}
\put(-2.25,-4){\line(1,0){180}}
\linethickness{0.5mm}%% Linienbreite der Aussenlinien
\put(178,0){\line(0,-1){\value{doppeltextfeldhoehe}}}
\put(-2,0){\line(0,-1){\value{doppeltextfeldhoehe}}}
\put(-2.25,-\value{doppeltextfeldhoehe}){\line(1,0){180.5}}


%% Weiter:
%% schrittweite nicht als integer, sondern genau berechnen lassen wie?
%% fuer punkte und noten soviel platz lassen, wie gebraucht wird
%% notenberechnung durchfuehren und damit y ersetzen
%% eingabe fuer bei wieviel punkten gibt es eine 6 vorsehen
%% befehl im vorspann, der dieses notenliste mit entsprechenden parametern aufruft

%%% Local Variables: 
%%% mode: plain-tex
%%% TeX-master: t
%%% End: 
 %% Fuegt Punkte-Noten-Liste hinzu /in arbeit
}


%% Technologie:
\ifthenelse{\value{punktet}=0}{}%
{% 
%\begin{picture}(180,8.7)
\begin{picture}(180,8.7)
  \kaendefeld
  \put(0,3){Technologie}
  \put(128,3){Von \arabic{punktet} erreicht:}
\end{picture}
}%


\ifthenelse{\value{notenliste}=0}{
}{%
\begin{picture}(180,8.7)
  %%%%%%%%%%%%%%%%%%%%%%%%%%%%%%%%%%%%%%%%%%%%%%%%%%%%%%%%%%%%%%%%%%%%%%%%%%%%%%%%
%% Punkte-Notenliste einfuegen:
%\newcounter{pnpunkte}

%\newcounter{notenbox}% Anzahl der Notenboxen
%\newcounter{notenboxort}% Wo steht die Notenbox
%\newcounter{schritt}% Breite der Notenboxen
%\newcounter{gesamtpunktzahlplus}
%\newcounter{gesamtpunktzahlplusplus}
%\newcounter{textfeldhoehe}
\setcounter{textfeldhoehe}{4}
\setcounter{pnpunkte}{4}

%\newcounter{doppeltextfeldhoehe}
\setcounter{doppeltextfeldhoehe}{\value{textfeldhoehe}} 
\addtocounter{doppeltextfeldhoehe}{\value{textfeldhoehe}}

\setcounter{gesamtpunktzahlplus}{\value{gesamtpunktzahl}}
\setcounter{gesamtpunktzahlplusplus}{\value{gesamtpunktzahl}}
\addtocounter{gesamtpunktzahlplus}{1}
\addtocounter{gesamtpunktzahlplusplus}{2}
\setcounter{schritt}{178/\real{\value{gesamtpunktzahlplus}}}
\setcounter{notenboxort}{178} % Erste Notenbox ganz rechts aussen
\addtocounter{notenboxort}{-\value{schritt}}
\addtocounter{gesamtpunktzahlplus}{1}
\linethickness{0.2mm}%% Linienbreite der Zwischenlinien
\whiledo{\value{notenbox}<\value{gesamtpunktzahlplus}}% Test
{% Schleifentext Anfang
\put(\value{notenboxort},0){\line(0,-1){\value{doppeltextfeldhoehe}}}
\put(\value{notenboxort},-\value{textfeldhoehe}){\makebox(\value{schritt},\value{textfeldhoehe}){\scriptsize \thepnpunkte}}
\put(\value{notenboxort},-\value{doppeltextfeldhoehe}){\makebox(\value{schritt},\value{textfeldhoehe}){\scriptsize y}}
\addtocounter{notenboxort}{-\value{schritt}}
\addtocounter{notenbox}{1}
\addtocounter{pnpunkte}{1}
}% Schleifentext Ende
\put(-1,-\value{textfeldhoehe}){\makebox(8,\value{textfeldhoehe})[l]{\scriptsize
    Punkte}}
\put(-1,-\value{doppeltextfeldhoehe}){\makebox(8,\value{textfeldhoehe})[l]{\scriptsize Note}}
\linethickness{0.2mm}%% Linienbreite der Zwischenlinien
\put(-2.25,-\value{textfeldhoehe}){\line(1,0){180}}
\put(-2.25,-4){\line(1,0){180}}
\linethickness{0.5mm}%% Linienbreite der Aussenlinien
\put(178,0){\line(0,-1){\value{doppeltextfeldhoehe}}}
\put(-2,0){\line(0,-1){\value{doppeltextfeldhoehe}}}
\put(-2.25,-\value{doppeltextfeldhoehe}){\line(1,0){180.5}}


%% Weiter:
%% schrittweite nicht als integer, sondern genau berechnen lassen wie?
%% fuer punkte und noten soviel platz lassen, wie gebraucht wird
%% notenberechnung durchfuehren und damit y ersetzen
%% eingabe fuer bei wieviel punkten gibt es eine 6 vorsehen
%% befehl im vorspann, der dieses notenliste mit entsprechenden parametern aufruft

%%% Local Variables: 
%%% mode: plain-tex
%%% TeX-master: t
%%% End: 

\end{picture}
}



%% Mathe:
\ifthenelse{\value{punktem}=0}{}%
{%
\begin{picture}(180,8.7)
  \kaendefeld
  \put(0,3){Mathe}
  \put(128,3){Von \arabic{punktem} erreicht:}
\end{picture}
}%

%% Arbeitsplanung:
\ifthenelse{\value{punkteap}=0}{}%
{%
\begin{picture}(180,8.7)
  \kaendefeld
  \put(0,3){Arbeitsplanung}
  \put(128,3){Von \arabic{punkteap} erreicht:}
\end{picture}
}
%% EOF %%%%%%%%%%%%%%%%%%%%%%%%%%%%%%%%%%%%%%%%%%%%%%%%%%%%%%%%%%%%%%%%%%%%%%%%%
















%%% Local Variables: 
%%% mode: latex
%%% TeX-master: t
%%% End: 
