%^^A Diese Datei wird bei der Angabe der Option 'ka' in der 
%^^A Dokumentklasse 'teacher' eingelesen.
%^^A Es dient dazu die Formatierung der erzeugten Klassenarbeit zu steuern

% \DescribeMacro{loesungslinienzeigen} Zeigt Lösungslinien (1) 
% oder verbirgt sie (0)


% \DescribeMacro{teilaufnummerierung} Art der Nummerierung der 
% Aufgaben: a,b,c... (=0) oder 1.1,1.2,1.3...(=1)

% \DescribeMacro{aufgabenpunkte} Punktzahl zeigen = (1), verbergen = (0). 
% Gemeint ist die Punktzahl am Ende einer ganzen  Aufgabe.

% \DescribeMacro{aufgabenkopfzeile}
% Schaltet die Ausgabe der Aufgaben-Kopfzeile ein = (1), aus = (0).

% \DescribeMacro{aufgabennummerierung}
% Schaltet die Nummerierung der Aufgaben auf T :(1), oder auf Aufgabe :(0).

% \DescribeMacro{aufgabenstellung}
% Aufgabenstellung (Fragen) zeigen = (1), verbergen = (0).

% \DescribeMacro{aufgabenfusszeile}
% Schaltet die Ausgabe der Aufgaben-Fusszeile ein = (1), aus = (0).

% \DescribeMacro{loesungkopfzeile}
% Schaltet die Ausgabe der Lösungs-Kopfzeile ein = (1), aus = (0).

% \DescribeMacro{loesungen}
% Lösungen (Antworten) zu den Aufgaben zeigen = (1), verbergen = (0).

% \DescribeMacro{xlsg}
% Farbe des Lösungstextes auf grün = (1), verbergen (Weiß) = (0).

% \DescribeMacro{dateinamen}
% Dateinamen in der Aufgaben-Kopfzeile zeigen = (1), verbergen = (0).

% \DescribeMacro{dehnen}
% Variable Zwischenabstände ja = (1), nein = (0) (nicht vollständig
% implementiert)

% \DescribeMacro{fachangabe}
% Fach der Aufgabe im Aufgabentitel-Kopf zeigen = (1), oder aus = (0)

% \DescribeMacro{fachangabepzk}
% Fachangabe-Buchstaben (M, T, AP) am Punktzahl-Kasten 
% zeigen = (1), oder aus = (0)

% \DescribeMacro{punkteangabepzk}
% Erreichbare Punktezahl der Teilaufgabe am Punktzahl-Kasten 
% zeigen = (1), oder aus = (0)

% \DescribeMacro{punktzahlkasten}
% Punktzahl-Kasten zeigen = (1), oder aus = (0)

% \DescribeMacro{punktesummezeigen}
% Punktesumme der Teilaufgaben am Ende der Aufgaben
% zeigen = (1), oder aus = (0). Veraltet.

% \DescribeMacro{aufgabentitel}
% Name der Aufgabe hinter der Aufgabenmarke im Aufgabentitel-Kopf
% zeigen = (1), oder aus = (0)

% \DescribeMacro{gruppeninfo}
% Informationen über die Gruppen in der Aufgaben-Fusszeile zeigen.
% aus = (0), SW = (1) oder in Farbe = (1) ?????
% Farbe/SW wird mit bw

% \DescribeMacro{gruppeninfohead}
% Informationen über die Gruppen im Aufgaben-Kopf zeigen.
% aus = (0), SW = (1)

% \DescribeMacro{notenliste}
% Erzeugen einer Noteliste am Ende des Dokuments.
% aus = (0), zeigen = (1)

% \DescribeMacro{lkaabstand}
% Längenangabe.
% Zusatzabstand der Lösungslinien für den lka-Befehl.
% kommt zum Zeilenumbruch noch hinzu???

% \DescribeMacro{zusatzlkaabstand}
% Längenangabe.
% Zusätzlich zur Länge lkaabstand vor der ersten Lösungslinie eingefügt, 
% um Aufgabe und Lösung klarer zu trennen.

% ^^A ==========================================================================
% ^^A ==========================================================================
% ^^A ==========================================================================
% \StopEventually{}
% ^^A ==========================================================================
% ^^A ==========================================================================
% ^^A ==========================================================================

%    \begin{macrocode}
\typeout{Loading ka-format.tex}
\setcounter{loesungslinienzeigen}{1}
\setcounter{teilaufnummerierung}{0}
\setcounter{aufgabenpunkte}{0}
\setcounter{aufgabenkopfzeile}{1}
\setcounter{aufgabennummerierung}{0}
\setcounter{aufgabenstellung}{1}
\setcounter{aufgabenfusszeile}{0}
\setcounter{loesungkopfzeile}{0}
\setcounter{loesungen}{0}
\setcounter{xlsg}{0}
\setcounter{dateinamen}{0} 
\setcounter{dehnen}{0}
\setcounter{fachangabe}{0}
\setcounter{fachangabepzk}{0}
\setcounter{punkteangabepzk}{1}
\setcounter{punktzahlkasten}{1}
\setcounter{punktesummezeigen}{0}
\setcounter{aufgabentitel}{1}
\setcounter{gruppeninfo}{0}
\setcounter{gruppeninfohead}{0}
\setcounter{notenliste}{1}%%
\setlength{\lkaabstand}{3mm}
\setlength{\zusatzlkaabstand}{1mm}
%    \end{macrocode}

% lsg-Farbe auf weiss setzen
\definecolor{xlsgcolor}{gray}{1}
\definecolor{xlsgmustercolor}{gray}{1}

\endinput
