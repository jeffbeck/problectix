%^^A Diese Datei wird bei der Angabe der Option 'alles' in der 
%^^A Dokumentklasse 'teacher' eingelesen.
%^^A Es dient dazu die Formatierung zu steuern

\setcounter{loesungslinienzeigen}{1}
\setcounter{teilaufnummerierung}{0}
\setcounter{aufgabenpunkte}{0}
\setcounter{aufgabenkopfzeile}{1}
\setcounter{aufgabenstellung}{1}
\setcounter{aufgabenfusszeile}{1}
\setcounter{loesungkopfzeile}{1}
\setcounter{loesungen}{1}
\setcounter{dateinamen}{1} 
\setcounter{dehnen}{0}
\setcounter{fachangabe}{1}
\setcounter{fachangabepzk}{1}
\setcounter{punkteangabepzk}{1}
\setcounter{punktzahlkasten}{1}
\setcounter{punktesummezeigen}{1}
\setcounter{aufgabentitel}{1}
\setcounter{gruppeninfo}{2}
\setlength{\lkaabstand}{3mm}
\setlength{\zusatzlkaabstand}{1mm}
\setlength{\loesungslinienbreite}{0.35mm}
\setcounter{debuggen}{0}










