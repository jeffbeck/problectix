\begin{aufgabe}[Technologie]{Rohrleitungsmontage}
   \begin{textonly}
       F�r die Verbindung von Rohrleitungen aus Kupfer wird u. a. Silberlot
       verwendet. Das L�ten ist dabei nur unter Verwendung von Flussmitteln
       m�glich.
   \end{textonly}
   \begin{teilaufgabe}{t}{1}{1}
       Welches ist die Hauptaufgabe des Flussmittels?
   \end{teilaufgabe}
   \begin{loesung}
       \punkte{Oxidationsschicht entfernen}{1}{}
   \end{loesung}
   \begin{teilaufgabe}{t}{3}{3}
       Bei unsachgem��er L�tung kann es zur Bildung einer \glqq Klebenaht\grqq{}
       kommen.
    
       Was versteht man unter einer \glqq Klebenaht\grqq? Unter welchen
       Vorraussetzungen kommt sie zustande?
   \end{teilaufgabe}
   \begin{loesung}
       \punkte{Verbindung zu geringer Festigkeit}{1}{} 
       \punkte{,da keine Legierungsbildung Lot -- Werkstoff}{1}{} 
       \punkte{Arbeitstemperatur wurde nicht erreicht}{1}{}
   \end{loesung}
   \begin{teilaufgabe}{t}{2}{2}
       Wie kann w�hrend des L�tvorgangs die Zunderbildung im Rohrinnern
       verhindert werden?
   \end{teilaufgabe}
   \begin{loesung}
       \punkte{Rohrinneres mit Stickstoff oder Formiergas sp�len}{2}{}
   \end{loesung}
    \begin{teilaufgabe}{t}{2}{2}
       Welche 2 Aufgaben hat das Evakuieren der Anlage vor dem Bef�llen?
   \end{teilaufgabe}
   \begin{loesung}
       \punkte{Entfernen von Fremdgasen}{1}{}
       \punkte{und Wasser}{1}{}
   \end{loesung}
   \begin{teilaufgabe}{t}{1}{1}
       Kupferrohre k�nnen auch ohne Flussmittel verl�tet werden, wenn bestimmte
       Lote verwendet werden. Welcher Stoff muss in diesen Loten enthalten sein?
   \end{teilaufgabe}
   \begin{loesung}
       \punkte{Phosphor}{1}{}
   \end{loesung}
\end{aufgabe}



%%% Local Variables: 
%%% mode: latex
%%% TeX-master: t
%%% End: 
