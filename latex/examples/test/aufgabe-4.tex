%%%%%%%%%%%%%%%%%%%%%%%%%%%%%%%%%%%%%%%%%%%%%%%%%%%%%%%%%%%%%%%%%%%%%%%%%%%%%%%%
\begin{aufgabe}[Technologie]{Messabweichungen}
\begin{teilaufgabe}[ohnenummer]{t}{3}{2}
    Sie sollen mit dem Messschieber an einer schlecht zugänglichen Stelle einer
    Kälteanlage messen. Dabei können Sie nicht senkrecht auf die Skala blicken.
    Wie können Sie vermeiden, dass sie einen Messfehler durch \textit{Parallaxe}
    verursachen?
\end{teilaufgabe}
\begin{loesung}
    \punkte{Mit der Feststellschraube feststellen, abziehen, ablesen}{2}{}
\end{loesung}
\end{aufgabe}
