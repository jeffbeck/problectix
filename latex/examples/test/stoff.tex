\documentclass[stoff]{teacher}

\Name{Müller}
\Datum{22. 3. 01}
\Klasse{MKB}
\Blatt{17-A}
\UserToken{MyName}
\Fach{Technologie}
\Titelo{Ausgabeoption}
\Titelu{stoff}
\Revision{}
\Ausdruck{}

\begin{document}
\titel
Gesamtstunden: 11 Wochen zu je 4 Stunden = \textbf{44} Unterrichtsstunden

\begin{stoff}
  \thema{Organisatorisches}
    \stunde{-}{}{Einführung}{Namen, Sitzliste}{}
    \stunde{-}{}{Einführung}{Namen, Versicherungen, Stundenplan}{}
    \stunde{-}{}{Bücherausgabe}{}{}
    \stunde{-}{}{Noten --- Transparenz}{}{}
  \thema{Befestigungstechnik}
    \stunde{T}{}{Dübelarten}{}{}
    \stunde{M}{22.09.2004}{Kräfte, Gewichtskraft -- Übungen}{}{1.3.4}
    \stunde{M}{27.09.2004}{Hebel --- Grundlagen}{}{1.3.5}
    \stunde{M}{}{Hebel --- Übungen}{}{}
    \stunde{AP}{}{Drehmoment --- Grundlagen}{}{}
    \stunde{AP}{}{Drehmoment --- Übungen}{}{}
    \stunde{T}{}{Rohrverlegung}{}{}
    \klassenarbeit
    \block
  \thema{Energie und Leistung}
    \stunde{M}{}{mechanische Energie}{}{}
    \stunde{M}{}{mechanische Leistung}{}{}
    \stunde{M}{}{Übungen}{}{}
    \stunde{M}{}{Wirkungsgrad}{}{}
    \stunde{M}{}{Übungen}{}{}
  \thema{Berechnen von Rohrleitungen}
    \stunde{M}{}{Flächenberechnungen}{}{}
    \stunde{M}{}{Flächenberechnungen --- Übungen}{}{}
    \stunde{M}{}{Fließgeschwindigkeit}{}{}
    \stunde{M}{}{Fließgeschwindigkeit --- Übungen}{}{}
    \stunde{M}{}{Massenstrom, Volumenstrom}{}{}
    \stunde{M}{}{Massenstrom, Volumenstrom --- Übungen}{}{}
\end{stoff}
\end{document}













