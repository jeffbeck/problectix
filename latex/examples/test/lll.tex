\documentclass[arb,arblsg,bszleo,frame]{teacher}

\Name{M�ller}
\Datum{22. 3. 01}
\Klasse{MKB}
\Blatt{17-A}
\UserToken{MyName}
\Fach{Technologie}
\Titelo{Ausgabeoption}
\Titelu{arb,bszleo,frame}
\Quelle{Bronstein 1818}
\Revision{}
\Ausdruck{}

\begin{document}

\arbsection{Test eines L�ckentexts mit \texttt{\textbackslash lll}}

\arbsubsection{1. Test: Ohne L�ngenangaben bzw. optionale Optionen.}

Ein \lll{Haus} ist ein \lll{Baum} ist ein \lll{Schiff} ist eine
\lll{Insel} ist ein \lll{Meer} und somit auch wieder ein \lll{Haus}!

\arbsubsection{dito. mit zus�tzlichem Zeilen-Abstand}


\lue Ein \lll{Haus} ist ein \lll{Baum} ist ein \lll{Schiff} ist eine
\lll{Insel} ist ein \lll{Meer} und somit auch wieder ein \lll{Haus}!

\arbsubsection{2. Test: Mit L�ngenangaben bzw. optionale Optionen.}

Ein \lll[27]{Haus} ist ein \lll[27]{Baum} ist ein \lll[27]{Schiff} ist eine
\lll[27]{Insel} ist ein \lll[27]{Meer} und somit auch wieder ein \lll[27]{Haus}!

\arbsubsection{dito. mit zus�tzlichem Zeilen-Abstand}


\lue Ein \lll[27]{Haus} ist ein \lll[27]{Baum} ist ein \lll[27]{Schiff} ist eine
\lll[27]{Insel} ist ein \lll[27]{Meer} und somit auch wieder ein \lll[27]{Haus}!


\newpage


\arbsection{Test eines L�ckentexts mit \texttt{\textbackslash lllnumbered}}

\arbsubsection{1. Test: Ohne L�ngenangaben bzw. optionale Optionen.}

Ein \lllnumbered{Haus} ist ein \lllnumbered{Baum} ist ein
\lllnumbered{Schiff} ist eine \lllnumbered{Insel} ist ein
\lllnumbered{Meer} und somit auch wieder ein \lllnumbered{Haus}!

\arbsubsection{dito. mit zus�tzlichem Zeilen-Abstand}


\lue Ein \lllnumbered{Haus} ist ein \lllnumbered{Baum} ist ein
\lllnumbered{Schiff} ist eine \lllnumbered{Insel} ist ein
\lllnumbered{Meer} und somit auch wieder ein \lllnumbered{Haus}!

\arbsubsection{2. Test: Mit L�ngenangaben bzw. optionale Optionen.}

Ein \lllnumbered[27]{Haus} ist ein \lllnumbered[27]{Baum} ist ein
\lllnumbered[27]{Schiff} ist eine \lllnumbered[27]{Insel} ist ein
\lllnumbered[27]{Meer} und somit auch wieder ein
\lllnumbered[27]{Haus}!

\arbsubsection{dito. mit zus�tzlichem Zeilen-Abstand}


\lue Ein \lllnumbered[27]{Haus} ist ein \lllnumbered[27]{Baum} ist ein
\lllnumbered[27]{Schiff} ist eine \lllnumbered[27]{Insel} ist ein
\lllnumbered[27]{Meer} und somit auch wieder ein
\lllnumbered[27]{Haus}!

\bigskip

{\Large Hier sind alle Verwendetete Texte in der richtigen Reihenfolge:}

\lllliste[0]

\newpage


\arbsection{Verwendung von \texttt{\textbackslash lll} in der
  \texttt{tabbing}-Umgebung}

\arbsubsection{Ohne Nummerierung}


\lue
\begin{tabbing} 
Beobachtung:\hspace{8mm} \= \lll[138][l]{Wasser verdampft} \\
                   \> \lll[138][l]{Die Temperatur bleibt gleich} \\
                   \> \lll[138]{} \\
\\
Text zu 1:   \> \lll[138][s]{Text wird mit Option s auf der Linie verteilt} \\
Text zu 2:   \> \lll[138][s]{nocheinmal option s} \\
Text zu 3:   \> \lll[138][r]{Rechtsb�ndig auf der Linie mit Option r} \\
Text zu 4:   \> \lll[138][l]{Linksb�ndig auf der Linie mit Option l} \\
Text zu 5:   \> \lll[138][c]{Zentriert auf der Linie mit Option c} \\
\end{tabbing}

\arbsection{Benennung eines Bildes in dem Bauteile mit Ziffern versehen sind}

\lue
\begin{tabbing} 
\hspace{3mm}\= \hspace{1mm} \= \hspace{52mm} \=
\hspace{3mm}\= \hspace{1mm} \= \hspace{52mm} \=
\hspace{3mm}\= \hspace{1mm} \= \hspace{52mm}\\ \kill
\textbf{1}\>:\>\lll[46][l]{Geh�use}\>
\textbf{2}\>:\>\lll[46][l]{Zylinder}\>
\textbf{3}\>:\>\lll[46][l]{Kolben}\\
\textbf{4}\>:\>\lll[46][l]{Einlassventil}\>
\textbf{5}\>:\>\lll[46][l]{Auslassventil}\>
\textbf{6}\>:\>\lll[46][l]{Ventilplatte}\\
\textbf{7}\>:\>\lll[46][l]{Pleul}\>
\textbf{8}\>:\>\lll[105][l]{Verdichter mit Saugdampfk�hlung}\>
\\

\end{tabbing}




\arbsection{Fortsetzungstext}

Das ist der hinleitende Satz, der \lll[v][l]{von von den Sch�lern auf
  diese Zeilen weitergef�hrt}

\lll[v][l]{bzw. erg�nzt werden soll. Er kann auch ausgeblendet werden}

\lll[v][c]{\scriptsize (Das wurde in verringerter Gr��e geschrieben)}






\end{document}













