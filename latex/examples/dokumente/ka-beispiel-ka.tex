\documentclass[11pt,ka]{teacher}
%\Name{Deutsch}
\Fach{m}
%\Datum{22. 3. 01}
%\Klasse{BVJA}
%%\Blatt{17-A}
\Titelo{Übungsblatt}
\Titelu{Präteritum}
%\Quelle{Bronstein 1818}
%\leer
%%\arb[nb]
%%\arbrahmen[nb]
%%\kaarb[nb]
%\aufgabensammlung
\gruppea
%%\loesungas
\newcommand{\aufundlsg}{
\setcounter{aufgabenkopfzeile}{1} \setcounter{aufgabenstellung}{1}
\setcounter{aufgabenfusszeile}{1} \setcounter{loesungkopfzeile}{1}
\setcounter{loesungen}{1} \setcounter{teilaufnummerierung}{0}}
%\aufundlsg %Aufgabe und Lösungen zeigen
%\auf % Nur Aufgaben zeigen
%\lsg % Nur Lösungen zeigen
%% 1.1, 1.2(=1) oder a) b) (=0)
%%\setcounter{teilaufnummerierung}{0}

\begin{document}
%%\kaarb
%%\klassenarbeit
%%\ohne{1}
\begin{aufgabe}[Mathe]{Eignungsaufgabe}
\begin{teilaufgabe}{m}{5}{9} \liniert
    Wie heißt die Hauptstadt von Frankreich?
\end{teilaufgabe}
    \begin{loesung}
        \punkte{Paris}{1}{Bordeaux zählt halber Punkt}
    \end{loesung}
\begin{teilaufgabe}{ap}{4}{3} \kariert
    Wer war Nero?
\end{teilaufgabe}
    \begin{loesung}
        \punkte{Kaiser}{4}{Politiker zählt halber Punkt}
    \end{loesung}
\end{aufgabe}

% \begin{aufgabe}[Technologie]{Probeaufgabe}
%    \begin{teilaufgabe}{m}{8}{3}
%       Wo gehts lang?
%    \end{teilaufgabe}
%    \begin{loesung}
%        \punkte{Hier}{2}{Alles andere ist Falsch!}
%    \end{loesung}
% \end{aufgabe}



\nehme{aufgabe-1}
\nehme{aufgabe-2}
\nehme{aufgabe-3}
\nehme{aufgabe-4}
\nehme{aufgabe-5}

\ohne[-1]{1}

\nehme{aufgabe-6}
\ohne[-2]{2}

\nehme{aufgabe-6}

\end{document}













