\documentclass{teacher}
%%\usepackage{teacherpack} %% wird von der dokumentklasse 'teacher' geladen
\Name{R�diger Beck}
\Fach{m}
\Datum{22. 3. 01}
\Klasse{BVJA}
%%\Blatt{17-A}
\Titelo{�bungsblatt}
\Titelu{Pr�teritum}
\Quelle{Bronstein 1818}

%% folgendes sollte als optionen in der documentclass angebbar sein
%%\leer
%%\arb[nb]
\arbrahmen[nb]
%%\kaarb[nb]



\begin{document}
%%\klassenarbeit
Testaussehen \theformatierung
\setlength{\lkaabstand}{9mm}

\lka[3]

\grad und das 

\entspricht{} gleich 
fgf gf hjf fghf fghfgf f 

\wort[0mm]{test}{her is a beispiek}{check}
\wort[0mm]{test}{her is a beispiek}{check} 
\wort[0mm]{test}{her is a beispiek}{check} 

itzrhj jpi uiph hgzi g gftft ft ft f tftizf tzft f tzif



\bspsatz[Was tun]{Beispielsatz}{Antwort}

\bspzweisatz[Was tun]{Beispielsatz}{Frage 1:}{Antwort 1}{Frage 2:}{Antwort 2}

\normalsize


t zg zgo ug zg \frage{Was is hier los} hjh hjhjk hkjh jkh iu uh hui hu ihiuhjk hkjh jkh iu uh hui hu ihiuhjk hkjh jkh iu uh hui hu ihiuhjk hkjh jkh iu uh hui hu ihiukh \wort[0mm]{test}{her is a beispiek}{check} hjk hkjh jkh iu uh hui hu ihiu
huhihu huihuih
 \llv[4]
Weiter
\unterschrift




Testaussehen 
\grad und das 

\entspricht{} gleich 
fgf gf hjf fghf fghfgf f 

\wort[0mm]{test}{her is a beispiek}{check}
\wort[0mm]{test}{her is a beispiek}{check} 
\wort[0mm]{test}{her is a beispiek}{check} 

itzrhj jpi uiph hgzi g gftft ft ft f tftizf tzft f tzif

\bspsatz[Was tun]{Beispielsatz}{Antwort}


Testaussehen 
\grad und das 

\entspricht{} gleich 
fgf gf hjf fghf fghfgf f 

\wort[0mm]{test}{her is a beispiek}{check}
\wort[0mm]{test}{her is a beispiek}{check} 
\wort[0mm]{test}{her is a beispiek}{check} 

itzrhj jpi uiph hgzi g gftft ft ft f tftizf tzft f tzif

\bspsatz[Was tun]{Beispielsatz}{Antwort}


Testaussehen 
\grad und das 

\entspricht{} gleich 
fgf gf hjf fghf fghfgf f 

\wort[0mm]{test}{her is a beispiek}{check}
\wort[0mm]{test}{her is a beispiek}{check} 
\wort[0mm]{test}{her is a beispiek}{check} 

itzrhj jpi uiph hgzi g gftft ft ft f tftizf tzft f tzif

\bspsatz[Was tun]{Beispielsatz}{Antwort}


itzrhj jpiitzrhj jpi uiph hgzi g gftft ft ft f
tftizf tzft f tzifitzrhj jpi uiph hgzi g gftft ft ft f tftizf tzft f tzifitzrhj
jpi uiph hgzi g gftft ft ft f tftizf tzft f tzifitzrhj jpi uiph hgzi g gftft ft
ft f tftizf tzft f tzifitzrhj jpi uiph hgzi g gftft ft ft f tftizf tzft f tzif
uiph hgzi g gftft ft ft f tftizf tzft f tzif 



itzrhj jpiitzrhj jpi uiph hgzi g gftft ft ft f
tftizf tzft f tzifitzrhj jpi uiph hgzi g gftft ft ft f tftizf tzft f tzifitzrhj
jpi uiph hgziitzrhj jpiitzrhj jpi uiph hgzi g gftft ft ft f
tftizf tzft f itzrhj jpiitzrhj jpi uiph hgzi g gftft ft ft f
tftizf tzft f tzifitzrhj jpi uiph hgzi g gftft ft ft f tftizf tzft f tzifitzrhj
jpi uiph hgzi g gftft ft ft f tftizf tzft f tzifitzrhj jpi uiph hgzi g gftft ft
ft f tftizf tzft f tzifitzrhj jpi uiph hgzi g gftft ft ft f tftizf tzft f tzif
uiph hgzi g gftft ft ft f tftizf tzft f tziitzrhj jpiitzrhj jpi uiph hgzi g gftft ft ft f
tftizf tzft f tzifitzrhj jpi uiph hgzi g gftft ft ft f tftizf tzft f tzifitzrhj
jpi uiph hgzi g gftft ft ft f tftizf tzft f tzifitzrhj jpi uiph hgzi g gftft ft
ft f tftizf tzft f tzifitzrhj jpi uiph hgzi g gftft ft ft f tftizf tzft f tzif
uiph hgzi g gftft ft ft f tftizf tzft f tzitzifitzrhj jpi uiph hgzi g gftft ft ft f tftizf tzft f tzifitzrhj
jpi uiph hgzi g gftft ft ft f tftizf tzft f tzifitzrhj jpi uiph hgzi g gftft ft
ft f tftizf tzft f tzifitzrhj jpi uiph hgzi g gftft ft ft f tftizf tzft f tzif
uiph hgzi g gftft ft ft f tftizf tzft f tziitzrhj jpiitzrhj jpi uiph hgzi g gftft ft ft f
tftizf tzft f tzifitzrhj jpi uiph hgzi g gftft ft ft f tftizf tzft f tzifitzrhj
jpi uiph hgzi g gftft ft ft f tftizf tzft f tzifitzrhj jpi uiph hgzi g gftft ft
ft f tftizf tzft f tzifitzrhj jpi uiph hgzi g gftft ft ft f tftizf tzft f tzif
uiph hgzi g gftft ft ft f tftizf tzft f tzi g gftft ft ft f tftizf tzft f tzifitzrhj jpi uiph hgzi g gftft ft
ft f tftizf titzrhj jpiitzrhj jpi uiph hgzi g gftft ft ft f
tftizf tzft f tzifitzrhj jpi uiph hgzi g gftft ft ft f tftizf tzft f tzifitzrhj
jpi uiph hgzi g gftft ft ft f tftizf tzft f tzifitzrhj jpi uiph hgzi g gftft ft
ft f tftizf tzft f tzifitzrhj jpi uiph hgzi g gftft ft ft f tftizf tzft f tzif
uiph hgzi g gftft ft ft f tftizf tzft f tzizft f tzifitzrhj jpi uiph hgzi g gftft ft ft f tftizf tzft f tzif
uiph hgzi g gftft ft ft f tftizf tzft f tzi
Testaussehen 
\grad und das 

\entspricht{} gleich 
fgf gf hjf fghf fghfgf f 

\wort[0mm]{test}{her is a beispiek}{check}
\wort[0mm]{test}{her is a beispiek}{check} 
\wort[0mm]{test}{her is a beispiek}{check} 

itzrhj jpi uiph hgzi g gftft ft ft f tftizf tzft f tzif

\bspsatz[Was tun]{Beispielsatz}{Antwort}



\end{document}
















