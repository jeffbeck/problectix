
\documentclass{article}

\usepackage[latin1]{inputenc}

\begin{document}

\section{Einf�hrung in \LaTeX}

Grundaufbau, article

Frage formulieren: Vorteile Aluminium gegen�ber Kupfer.

�bersetzen

*.tex *.dvi *.ps

*.tex *.pdflatex 

xdvi nutzen

Zusatzoption [12pt]

-> Grundstruktur \befehl[optional]{parameter}
-> Grundstruktur Umgebung
   \begin[optional]{umgebungsname}
   \end{umgebungsname}


Frage mit Umlauten  : in welcher Beziehung ist Kupfer �berlegen

Frage mit Umlauten  : in welcher Beziehung ist Kupfer �berlegen

Fragen (Mathe) : Wie lautet der Formelzusammenhang zwischen dem
Volumen V, der Masse m und der Dichte rho.


Frage Erl�utern sie die Formelzeichen der folgenden Formel

     m c deltaT

Welchen Formelzusammenhang erl�utert folgende Formel

Mischungsformel (Bruch index 1,2 oder kalt,warm)

(evtl. Einheiten)




Aufgabe: Erstellen sie 2 Aufgaben, 1 Text incl. Umlaute, 1 mit Formel



\section{Die Dokumentklasse teacher}


\subsection{Aufgaben}

Kopfzeile, Fusszeilen, Blattbreite, Pakete laden ...

-> Vorlage teacher


Strukturieren

aufgabe

teilaufgabe


Strukturieren

Optionen von teacher



Aufgabe: Umsetzen it eigenen Aufgaben


\subsection{Zusatzoptionen}

Optionen:

option ka,kamulti

option bszleo, bszleoexam

option notenliste


\subsection{L�sungen}




\section{Datenbankaufbau}

Auslagern in andere Dateien

aufgabe1-bz.tex
aufgabe2-bz.tex



\section{Datenaustausch}





\section{Links}

\subsection{kile Entwicklungsumgebung}

http://www.linux-user.de/ausgabe/2005/03/060-kile/








\end{document}




%%% Local Variables: 
%%% mode: latex
%%% TeX-master: t
%%% End: 
